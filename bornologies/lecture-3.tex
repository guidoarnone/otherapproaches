
The equibounded bornology can be generalized to the context of multilinear maps. Namely, 
given $n \in \N$ and $V_1, \ldots, V_n, W$ bornological vector spaces, we say that a  
family $L$ of multilinear maps $V_1 \times \cdots \times V_n \to W$ is \emph{equibounded} if
for each collection of bounded subsets $B_i \in \mathcal B(V_i)$, $i \in \{1, \ldots, n\}$ the set
\[
  L(B_1 \times \cdots \times B_n) = \{T(B_1 \times \cdots \times B_n) : T \in L\}  
\]
is bounded on $W$. A multilinear map $T \colon V_1 \times \cdots \times V_n \to W$ is \emph{bounded}
if it maps products of bounded subsets to bounded subsets; equivalently $T$ is bounded if $\{T\}$ 
is equibounded. 
An argument along the lines of Lemma \ref{lem:equi-born} shows that
bounded multilinear maps together with subsets of equibounded
multilinear maps form a bornology; we denote this bornological vector space by
$\Hom^{(n)}(V_1 \times \cdots \times V_n; W)$.
We record the following consequence from the definitions for future usage.
\begin{lemma}\label{lem:bounded-coarsefine-mult}
Let $V_1, \ldots, V_n, W$ be bornological vector spaces and let
$T \colon V_1 \times \cdots \times V_n \to W$ be a bounded multilinear map.
Given bornologies $B_i \subset \mathcal B(V_i)$ 
for each $i \in \{1, \ldots,n\}$ and $B \supset \mathcal B(W)$, the map $T$ 
remains bounded as a multilinear map $(V_1, B_1) \times \cdots \times (V_n, B_n) \to (W,B)$.  
\qed
\end{lemma}

When the bornological vector spaces considered arise from 
locally convex topological vector spaces, 
a natural question to ask is 
what precisely is the relation between continuity and boundedness 
of a multilinear function. Not only that, one may wonder if there are differences
in the notions of boundedness 
when using the von Neumann bornology or the precompact bornology. 
As it turns out, if $V_1, \ldots, V_n$
are additionally Frechét speces then all of 
these notions are one and the same:

\begin{proposition}\label{prop:mult-frechet} Let $V_1, \ldots, V_n$ be Frechét spaces and let $W$ be 
a locally convex topological vector space.  Given an $n$-multilinear
map $T \colon V_1 \times \cdots \times V_n \to W$, 
the following statements are equivalent:
\begin{itemize}
    \item[(i)] The map $f$ is continuous.
    \item[(ii)] The map $f$ is separately continuous, that is, for each $i \in \{1, \ldots, n\}$
    and $v \in \prod_{j \neq i} V_j$, the map $f(v_1, \ldots, v_{i-1}, -, v_{i+1}, \ldots, v_n) \colon V_i \to W$ 
    is continuous. 
    \item[(iii)] Given sequences $(v^1_k)_{k \in \N}, \ldots, (v^n_k)_{k \in \N}$ converging to zero, 
    the sequence $(T(v^1_k, \ldots, v^n_k))_{k \in \N}$ is (von Neumann) bounded in $W$. 
    \item[(iv)] The map $f$ is bounded as a multilinear map $\Cpt(V_1) \times \cdots \times \Cpt(V_n) \to \Cpt(W)$.
    \item[(v)] The map $f$ is bounded as a multilinear map $\vN(V_1) \times \cdots \times \vN(V_n) \to \vN(W)$.
    \item[(vi)] The map $f$ is bounded as a multilinear map $\Cpt(V_1) \times \cdots \times \Cpt(V_n) \to \vN(W)$.
\end{itemize}
\end{proposition}
\begin{proof} It is a classical results that
conditions (i), (ii), and (iii) are equivalent to each other (see 
for instance \cite{??}*{??}). 
Conditions (iv) and (v) both imply (vi) by Lemma \ref{lem:bounded-coarsefine-mult}. 
To see that (vi) implies (iii) it suffices to show that a sequence $S = (v^i_k)_{k\in\N} 
\subset V_i$ converging to zero is a proecompact subset. Indeed, if 
$U$ is a neighbourhood of the origin, we know that there exists $n_0 \in \N$
such that $v^i_k \in U$ if $k \ge n_0$ and so 
\[
S \subset \{v^i_1, \ldots, v^i_{n_0}\} \cup U \subset \{0,v^i_1, \ldots, v^i_{n_0}\} + U.
\]
To conclude we note that (i) implies both (iv) and (v). Both assertions 
stem from the fact that products of precompact (resp. von Neumann bounded) subsets
are precompact (resp. von Neumann bounded) in $V_1 \times \cdots \times V_n$, and 
a continuous map $V_1 \times \cdots \times V_n \to W$ is bounded 
when equipping $V_1 \times \cdots \times V_n$ and $W$ with the precompact (resp. von Neumann)
bornology. 
\end{proof}


\subsection{Convergence}

We continue expanding our dictionaly of notions in the
setting of bornological vector spaces, comparing these 
definititons
with their classical counterparts in functional analysis.

Let $V$ be a bornological vector space. We say that 
a sequence $(x_n)_{n \in \N} \subset V$ is 
\emph{bornologically convergent} to $x \in V$ if there 
exists a bounded disk $D$ such that $(x_n)_{n \in \N}$ converges
to $x$ in $V_D$. Similarly, we say that $(x_n)_{n \in \N}$ 
is a bornological a Cauchy sequence if it 
is a Cauchy sequence in $V_D$ for some bounded disk $D$.

\begin{example} \label{ex:conv-semi}
If $V$ is a semi-normed space and we equip it with its
von Neumann bornology, the usual notions of convergence
and Cauchy sequences coincide with the bornological ones. 
\end{example}

\begin{lemma} \label{lem:finer-conv} Let $V$ a vector space and $B \subset B'$
two bornologies on $V$. If $(x_n)_{n \in \N} \subset V$ is 
bornologically convergent to $x \in V$ for $B$, then it 
is also bornologically convergent to $x \in V$ for $B'$.
\end{lemma}
\begin{proof} By hypothesis $x_n \to x$ in $V_D$
for some disk $D \in B$. This disk is also bounded for $B'$, 
from which the lemma follows.
\end{proof}

\begin{proposition}[Uniqueness of limits] A 
bornonological vector space $V$ is separated 
if and only if every sequence converges 
to at most one point. 
\end{proposition}
\begin{proof} Suppose that a sequence $(x_n)_{n \in \N}$ 
converges simlultaneously to two
elements $x,y \in V$. By definition, there exist 
bounded disks $D, D' \in \mathcal B_d(V)$ such that 
$x_n \to x$ in $V_D$ and $x_n \to y$ in $V_{D'}$. 
Given that $V$ is separated, there exists a 
norming disk $D''$ containing both $D$ and $D'$. 
Note that in particular we have continuous
inclusions $(V_D, p_D) \to (V_{D''}, p_{D''})$
and $(V_{D'}, p_{D'}) \to (V_{D''}, p_{D''})$.
It follows that $(x_n)_{n \in \N}$ converges to both 
$x$ and $y$ in $V_{D''}$. 
Since $V_{D''}$ is a normed space, it is Hausdorff; 
this implies that $x = y$ as wanted.

Conversely, suppose that convergent 
sequences in $V$ have unique limits. 
Given a bounded subset $B$, we will see that its
disked hull $D$ is normed. If it were not the case,
then there must exist $z \in V_D \setminus \{0\}$ such that 
the gauge semi-norm $p_D$
associated to $D$ satisfies 
$p_D(z) = 0$. This is absurd, as it would 
imply that the constant sequence $z_n := z$
converges both to $z$ and $0$. 
\end{proof}

Example \ref{ex:conv-semi} can be 
generalized to Frechét spaces, as the following result shows.

\begin{theorem} \label{thm:frechet-convergence}
Let $V$ be a Frechét space and $(x_n)_{n \in \N} \subset V$.
The following statements are equivalent:
\begin{itemize}
    \item[(i)] The sequence $(x_n)_{n \in \N}$ is convergent (resp. Cauchy).
    \item[(ii)] The sequence $(x_n)_{n \in \N}$ is bornologically convergent (resp. Cauchy)
    for the von Neumann bornology.
    \item[(iii)] The sequence $(x_n)_{n \in \N}$ is bornologically convergent (resp. Cauchy)
    for the precompact bornology.
\end{itemize}
\end{theorem}
\begin{proof} 
The fact that (ii) and (iii)
are equivalent statements
is left as Exercise \ref{ex:conv-Cpt=vN}; we prove that (i) 
is equivalent to (ii).

Suppose that $(x_n)_{n\in\N}$ 
is bornologically convergent for the von Neumann bornology to a point $x \in V$ and
let $\mathcal F = \{p_i\}_{i \in I}$ be a family of semi-norms defining the topology of $V$.
By definition, there exists a von Neumann bounded disk $D$ such that
$x_n \to x$ in $V_D$. Since $D$ is bounded 
the unit ball $B_i$ of each semi-norm $p_i$ 
absorbs $D$, which implies that $p_i(x_n) \to p(x)$ for all semi-norms $p_i \in \mathcal F$. 
Because the topology on $V$ is initial 
with respect to the family $\mathcal F$, 
it follows that $(x_n)_{n \in \N}$ 
converges to $x$ in $V$. The statement 
for Cauchy sequences is proved analogously.

Conversely, suppose that $X = (x_n)_{n \in \N}$ converges topologically to a point $x \in V$.
Without loss of generality we may suppose that the family of semi-norms 
defining the is increasing.
In particular, we may fix a countable set $(V_n)_{n\in\N}$ of 
circled, convex, decreasing set of neighbourhoods of $0$.
By hypothesis, for each $n \in \N$ all but finitely many elements of $X$
lie in $V_n$. Thus, upon dilating $V_n$ by a suitable positive constant $\lambda_n$,
we have that $X \subset \lambda_n V_n$.

Put $B = \cap_{n \ge 1} n\lambda_n V_n$. It follows that $B$ is a bounded disk; we shall presently 
see that $x_n \to x$ in $V_B$. It suffices to see that given $M \ge 1$,
all but finitely many elements of $X$ lie in $(1/M)B$. Note that
\[
(1/M)B
= \bigcap_{n \le M} (n/M) \lambda_n V_n 
\cap \bigcap_{n > M} (n/M) \lambda_n V_n.
\]
Since $X \subset \lambda_n V_n \subset (n/M)  \lambda_n V_n$ for all $n > M$, it remains 
to show that all but finitely many elements of $X$ lie in $\bigcap_{n \le M} n\lambda_n V_n$, which follows from the fact that this is a neighbourhood of
the origin and $x_n\to 0$ topologically.
The same argument replacing $X$ by
$Y := \{x_n-x_m : n,m \in\N\}$ yields the result for Cauchy sequences.
\end{proof}


%\begin{definition}[absolutely summable, continuous and smooth functions] Let $V$ be a bornological vector space and $X$ a set.
%A function $f \colon X \to V$ is said to be \emph{absolutely summable} if $p_D \circ f \colon X \to \R_{\ge 0}$ 
%is absolutely summable for some bounded disk $D \subset V$.
%The space of such functions will be denoted $\ell^1(X,V)$.

%Suppose additionally that $X$ is a compact
%topological space. A function $f \colon X \to V$ 
%is said to be
%continuous if there exists $D\in %\mathcal B_d(V)$
%such that it can be corestricted to a 
%continuous function $X \to V_D$. 
%The space of continuous will be denoted $\mathcal C(X,V)$. 

%A subset $S \subset \mathcal C(X,V)$ is \emph{uniformly bounded} if they the elements of $S$ can all be corestricted to $V_D$
%for some bounded disk $D$. We say that $S$ is \emph{uniformly
%continuous} if it is uniformly bounded in the sense above
%and additionally the family $\{f|^{V_D} : f \in S\}$ is uniformly continuous.
%\end{definition}

%\begin{proposition} If 
%$V$ is a Frechét space, 
%a series is absolutely summable
%if it is so for the von Neumann
%or precompact bornologies.
%\end{proposition}
%\begin{proof}
%\end{proof}

%The collections of uniformly bounded and 
%uniformly continuous maps endow 
%$\mathcal C(X,V)$ with
%two bornological vector space structures. 
%Unless explictly mentioned, we consider $\mathcal C(X,V)$
%with the uniformly continuous bornology. 

%\begin{example} If $X$ is a 
%compact topological space, 
%then $C(X, \C) = C(X)$ is a Banach
%space. By the Arzelà-Ascoli
%theorem, its bornology of 
%uniformly continuous subsets 
%coincides with the precompact bornology. In contrast, its
%bornology of uniformly bounded
%functions 
%is the von Neumann bornology.
%\end{example}

%\begin{theorem}[\cite{meyer-metrizable}*{Theorem 3.7}] Let $V$ be a Frechét
%space and $X$ a compact topological
%space. There is a bornological isomorphism
%\[
%\mathcal C(X, \Cpt(V)) \simeq \Cpt(\mathcal C(X,V)).
%\]
%Likewise, a subset of $\mathcal %C(X,V)$ is von Neumann bounded %if 
%and only if it is uniformly %bounded
%as a subset of $\mathcal C(X, %\vN(V))$.
%\qed
%\end{theorem}