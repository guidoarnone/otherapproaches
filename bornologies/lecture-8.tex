\subsection{Quasi-abelian categories}

In this section we fix an additive category  $\mathsf C$ 
with kernels and cokernels. 

Recall that the \emph{image} of a morphism $f \colon X \to Y$
is defined as $\im(f) := \ker(\coker(f))$, and its \emph{coimage}
as $\coim(f) := \coker(\ker(f))$.
There always exists a canonical comparison map between 
the coimage and the image, which fits in the following diagram:
\[
\begin{tikzcd}
\ker(f) \arrow{r} & X \arrow{d} \arrow{r} & Y \arrow{d}\arrow{r} & \coker(f)\\
& \coim(f) \arrow[dashed]{r}[above]{\exists!}& \im(f) & 
\end{tikzcd}
\]
We say that $f$ is \emph{strict} if the dashed arrow above is an isomorphism. Informally, 
a morphism is strict if it 
satisfies Noether's first isomorphism theorem. 

We say that $\mathsf C$ is \emph{abelian}
if all morphisms are strict. These are categories in which classical 
homological algebra takes place; important examples include that 
of the category of modules over a ring and sheaves 
of abelian groups over a space. None of the 
categories that we have considered so far are abelian. To see this, 
we first record the following lemma.

\begin{proposition}\label{prop:ab-epimono=iso} Let $f \colon X  \to Y$ be a strict morphism in $\mathsf C$. 
If $f$ is both a monomorphism and an epimorphism, then 
it is an isomorphism.
\end{proposition}
\begin{proof}
If $f$ is a monomorphism, then $\ker(f) = 0$. Likewise, the fact that $f$
is an epimorphisms implies that $\coker(f) = 0$. It follows that
$\coim(f) = \coker(0 \to X) = X$ and $\im(f) = \ker(Y \to 0)$; consequently, 
the comparison map $\coim(f) \to \im(f)$ can be identified with $f$ itself.
\end{proof}

\begin{corollary} The categories $\Ban$, $\Born_\C$, $\Born_\C^s$ and $\CBorn_\C$ 
are not abelian.
\end{corollary}
\begin{proof} By Proposition \ref{prop:ab-epimono=iso}, it suffices 
to exhibit an arrow in each of the categories in question which is both
a monomorphism and an epimorphism but not an isomorphism.

In $\Born_\C$, $\Born_\C^s$ and $\CBorn_\C$, 
we may take the identity function of an infinite dimensional
Banach space $X$ viewed as an arrow $\Cpt(X) \to \vN(X)$.
To see that this is not an isomorphism note
for example that the open unit ball of $X$ is von Neumann
bounded but not precompact.

For $\Ban$ we may consider the inclusion $\ell^1 \to c_0$. This map
is injective with dense range, and so it is both a monomorphism
and an epimorphism. However, 
it is not bijective and thus it cannot be an isomorphism.
\end{proof}

As the result above shows, if we wish
to do homological algebra in our setting 
we will need to consider a less restrictive
notion than that of an abelian category. 
We say that $\mathsf C$ is \emph{quasi-abelian}
if strict monomorphisms are stable under pushouts
and struct epimorphisms are stable under pullbacks.
Explictly, this means that for every square
\begin{equation}\label{diag:quasiab}
\begin{tikzcd}
X \arrow{r}{f} \arrow{d} & Y \arrow{d} \\
Y' \arrow{r}{g} & Z
\end{tikzcd}
\end{equation}
we have that:
\begin{itemize}
    \item[(i)] if \eqref{diag:quasiab} is a pullback square and $g$ is a strict epimorphism, then so is $f$.
    \item[(ii)] if \eqref{diag:quasiab} is a pushout square and $f$ is a strict monomorphism, then so is $g$.
\end{itemize}