\part{Bornologies}

As mentioned in the introduction, the category of Banach spaces \(\mathsf{Ban}\) is the smallest meaningful category in which functional analysis combines well with homological algebra. Specifically, it is a closed, symmetric monoidal category with respect to the completed projective tensor product and the Hom-space equipped with the operator norm.  However, the category of Banach spaces is not closed under, say, projective limits, so that important algebras such as \(\Cont^\infty(M)\) are not included in it. There are two ways of remedying this -- working with the \emph{pro-completion} and the \emph{ind-completion}. The former category includes complete, locally convex topological vector spaces, but still lacks an internal Hom functor that is right adjoint to the completed projective tensor product. What we therefore instead look at is the category \(\mathsf{Ind}(\mathsf{Ban})\) of inductive systems of Banach spaces. It is also worth noting that so far we have not fixed a base field over which we propose to do functional analysis -- this is manifestly deliberate. We wish to develop a framework that yields desirable categorical outcomes no matter what the base field or ring may be.  \textbf{Write Leitfaden for chapter.}


\chapter{Concrete models for bornologies}

The category of inductive systems \(\mathsf{Ind}(\mathsf{Ban})\), while having useful categorical properties that we will eventually see, has the drawback that it is not a concrete category. It turns out, however, that it is equivalent (in a sense that will be made precise) to a concrete category. This category can alternatively be described by objects that are vector spaces with additional structure, axiomatically referred to as ``bounded sets''. So far, this alternative characterisation has been explicitly worked out in the case where the base field is \(\R\), \(\C\) with their Euclidean topologies (colloquially referred to as ``Archimedean'' fields) and a complete, discrete valuation field \(\dvf\) (such as \(\Q_p\)) with its usual ``nonarchimedean'' valuation.  

\section{Bornological functional analysis in the archimedean case}


Throughout this section, we work with vector spaces over \(\R\) and \(\C\). We first start with an investigation of the internal structure of topological vector spaces. More concretely, if we start with a vector space \(V\) with a semi-norm \(\rho \colon V \to \R_{\geq 0}\), we can define a subset \[B \defeq \setgiven{x \in V}{\rho(x) \leq 1},\] called the \emph{unit ball} of \(\rho\). This subset satisfies the following properties:

\begin{enumerate}
    \item For all scalars \(\lambda \in \C\) with \(\abs{\lambda} \leq 1\), we have \(\lambda \cdot B \subseteq B\);
    \item For all \(x\), \(y \in B\), \(tx + (1-t)y \in B\) for all \(t \in [0,1]\);
    \item If for all \(r \in [0,1)\) we have \(rx \in B\), then \(x \in B\);
    \item \(\R_{\geq 0} \cdot B = V\).
\end{enumerate}

\begin{definition}
    A subset \(D \subseteq V\) of a vector space is called a \emph{disk} if it satisfies the first two properties above. We call a disk \textit{internally closed} if it satisfies the third property. A disk is said to be \emph{absorbent} if it satisfies the fourth property. 
\end{definition}


What is insightful is that every semi-norm can be recovered from an internally closed, absorbent disk, as the following theorem reveals:

\begin{theorem}
   Let \(D \subseteq V\) be an internally closed, absorbent disk. There exists a semi-norm \(\rho \colon V \to \R_{\geq 0}\) whose closed unit ball is \(D\). 
\end{theorem}

\begin{proof}
    The required semi-norm is given by 
    \[\rho \colon V \to \R_{\geq 0}, \qquad \rho(x) \defeq \inf \setgiven{\lambda \geq 0}{x \in \lambda D}.\]
    To begin with, this assignment is well-defined because if \(x \in V\), then since \(D\) is absorbent, there really is a \(\lambda \in \R_{\geq 0}\) and a \(v \in V\) such that \(x = \lambda v\). To see that it is scale-invariant, let \(\lambda \in \C\) and \(x \in V\). If \(\lambda = 0\), then this is clear, so let us assume otherwise. 
    Moreover, since $S^1 \cdot D = D$, it follows that $\rho(\omega x) = \rho(x)$ for all $\omega \in S^1$, $x \in V$
    and thus we may assume $\lambda \in \R_{> 0}$.
    We then have \(\rho(\lambda x) = \inf \setgiven{\alpha \geq  0}{\lambda x \in \alpha D} = \inf \setgiven{\alpha \geq 0}{x \in \frac{\alpha}{\lambda} D }\) and
    \begin{align*}
        \lambda \rho(x) &= \lambda\inf \setgiven{\alpha \geq 0}{x \in \alpha D} \\
        & =\inf \setgiven{\beta = \lambda\alpha }{x \in \alpha D} = \inf\left\{\beta \ge 0 :  x \in \frac{\beta}{\lambda} D \right\},
    \end{align*} as required. Finally, to show the triangle inequality, we use that \(D\) is absorbent to find \(\alpha\) and \(\beta\) such that \(x \in \alpha D\) and \( y \in \beta D\). Therefore, \(x + y \in \alpha D + \beta D = (\alpha + \beta) D,\) where the second equality follows from convexity. Consequently, \(\rho(x + y) \leq \alpha + \beta\). Since \(\alpha\) and \(\beta\) are arbitrary, we have \(\rho(x+ y) \leq \rho(x) + \rho(y)\). That the unit ball of this semi-norm is \(B\) is clear.
\end{proof}

The semi-norm in the proof of theorem above is called the \emph{gauge semi-norm} on a disk. Note that if \(D\) is a not necessarily absorbent disk in \(V\), then we can make it absorbent in the subspace \(V_D \defeq \R_{\geq 0} \cdot D\). The latter when equipped with the gauge semi-norm
 \[\rho \colon V \to \R_{\geq 0}, \qquad \rho(x) \defeq \inf \setgiven{\lambda \geq 0}{x \in \lambda D}.\]
becomes a semi-normed space inside \(V\), which in itself is just a vector space. We shall call a disk \(D\) \emph{norming} if \(V_D\) with the gauge semi-norm is actually a normed space; we call it \emph{complete} if \(V_D\) is completed and normed, that is, a Banach space. In conclusion, we have in some sense created a topological vector space from a disk, which is just a subset of a vector space with some axioms. This is exactly what we set out to do.


\begin{definition}[Bornology]
A \emph{bornology} on a vector space \(V\) is a collection \(\mathfrak{B}\) of its subsets (called \emph{bounded subsets}) satisfying the following natural properties:

\begin{itemize}
\item \(\{x\} \in \mathfrak{B}\) for all \(x \in V\);
\item if \(S \subseteq T\) and \(T \in \mathfrak{B}\), then \(S \in \mathfrak{B}\);
\item \(S \cup T \in \mathfrak{P}\) whenever \(S\), \(T \in \mathfrak{B}\);
\item if \(S \in \mathfrak{B}\), then \(r S \in \mathfrak{B}\) for \(r>0\);
\item any subset \(S \in \mathfrak{B}\) is contained in a disk \(T \in \mathfrak{B}\). 
\end{itemize}


\end{definition}

We call a bornological vector space \emph{separated} if every bounded subset is contained in a norming disk; we call it \emph{complete} if every bounded subset is contained in a complete, bounded disk. Denote by \(\mathcal{B}(V)\) the collection of bounded subsets of a bornological vector space \(V\). We also have two other related collections, namely, the collections \(\mathcal{B}_d(V)\) and \(\mathcal{B}_c(V)\) of bounded disks, and complete bounded disks. These collections carry two canonical preorders, namely, inclusion and \emph{absorption}, where we say \(S \subseteq_a T\) if there exists a \(c>0\) such that \( S \subseteq c T\). Recall that sets with preorders \((S, \leq)\) form a category, where objects are given by elements of the set, and there is a unique morphism \(x \to y\) if and only if \(x \leq y\). We want the preordered sets \(\mathcal{B}(V) \supseteq \mathcal{B}_d(V) \supseteq \mathcal{B}_c(V)\) to carry all the topological information in a topological vector space, in the same sense that a sequence of seminorms carries all the information about, say, a Frechet space. To make this more precise, we will need that these preordered sets viewed as categories are directed sets:

\begin{lemma}\label{lem:bornologies-directed}
The categories \(\mathcal{B}(V)\), \(\mathcal{B}_d(V)\) and \(\mathcal{B}_c(V)\) are directed. Furthermore, the subset \(\mathcal{B}_d(V)\) is cofinal in \(\mathcal{B}(V)\). The subset \(\mathcal{B}_c(V)\) is cofinal in \(\mathcal{B}(V)\) if and only if \(V\) is complete.  
\end{lemma}

\begin{proof}
    Let \(D_1\) and \(D_2\) be bounded subsets of \(V\). Then \(D_1 + D_2\) is contained in the convex hull of \(2D_1 \cup 2D_2\), which is bounded as bornologies are closed under multiplication by positive real numbers, taking finite unions and convex hulls. Consequently, \(D_1 + D_2\) is bounded, as bornologies are hereditary for inclusions. This shows that \(\mathcal{B}(V)\) is directed. To see that \(\mathcal{B}_d(V)\) is directed, we need that \(D_1 + D_2\) is a disk, whenever \(D_1\) and \(D_2\) are bounded disks, and this is easy to see. Finally, if \(D_1\) and \(D_2\) are complete, bounded disks, then we need to show that \(D_1 + D_2\) is complete. It suffices to show that \(V_{D_1 + D_2}\) is a Banach space. To see this, consider the canonical map \(q \colon V_{D_1} \oplus V_{D_2} \to V_{D_1 + D_2}\) taking \((x,y) \mapsto x + y\). Then there is an isometric isomorphism \(V_{D_1} \oplus V_{D_2}/\ker(q) \cong V_{D_1 + D_2}\), where the domain has the quotient norm. We leave the 
    details as Exercise \ref{ex:disk-sum-iso}.
\end{proof}


In the proof of Lemma \ref{lem:bornologies-directed}, we used that bornologies are closed under convex hulls. We also desire the same for bounded disks and complete, bounded disks. This is clear as \(\mathcal{B}_d(V)\) and \(\mathcal{B}_c(V)\) are both closed under arbitrary intersections (Exercise \ref{ex:disk-int}). We call the minimal disk (respectively, complete disk) containing a bounded subset (respectively, complete disk) the \emph{disked hull}, (respectively, \emph{complete disked hull}) 
of the given bounded subset.
Given any subset
$S \subset V$, its \emph{circled hull} is $S^{\circ} := \bigcup_{|\lambda| \le 1} \lambda S$. This is the minimal circled subset containing $S$. Likewise, 
recall that the convex hull of $S$ is 
the minimal convex set that contains $S$ and can be defined as $\conv(S) = \{\sum_{i=1}^n \lambda_i s_i : s_i \in S, \lambda_i \in \R_{>0}, \sum_{i=1}^n \lambda_i = 1\}$.
We can characterize 
the disked hull $S^\dhull$ of $S$ as
\begin{equation}\label{eq:Sd=Scc}
    S^\dhull = \conv(S^\circhull) = \left\{\sum_{i=1}^n \lambda_i s_i : s_i \in S, \lambda_i \in \C, \sum_{i=1}^n |\lambda_i| \le 1\right\}.
\end{equation}
The verification of the equality above
is the content of Exercise \ref{ex:circ-conv=disk}.

We can now finally look at some examples of bornologies.

\begin{example}
    The most basic example of a bornology is the \emph{fine bornology}. Its elements are subsets \(S \subseteq V\) that are contained in the disked hull of finite sets.  
\end{example}

It is sometimes convenient and economical to describe bornologies using a generating set of bounded subsets. To turn such a generating set into a bornology, we first notice that given a vector space, the power set \(\mathcal{P}(V)\) of \(V\) is always a bornology on \(V\). Consequently, if \(\mathcal{B}\) is a set of subsets of \(V\), we can always define the nonempty intersection 
\[ \gen{\mathcal{B}} \defeq \bigcap_{\mathcal{B}' \supseteq \mathcal{B}} \mathcal{B}'\] of 
bornologies containing \(\mathcal{B}\). This is itself a bornology, 
called the \emph{bornology generated by \(\mathcal{B}\)}; we leave the verification of this fact
as Exercise \ref{ex:int-bor}.

\begin{definition}\label{def:fine-coarse}
Given two bornologies $\mathcal B \subset \mathcal B'$, we say that $\mathcal B'$ is \emph{finer} than $\mathcal B$
or that  $\mathcal B$ is \emph{coarser} than $\mathcal B'$.
\end{definition}

Using the terminology defined above, the bornology generated by a set can be characterized as the coarsest 
bornology that contains it.


\begin{example}
    We now describe our first honest example of a bornology. Given a locally convex topological vector space \(V\) with a family of semi-norms \(p_i \colon V \to \R_{\geq 0}\), we define the \emph{von Neumann bornology} as the bornology where a subset \(B \subseteq V\) is bounded if \(p_i(B)\) is a bounded subset of \(\R\) for each \(i\). Equivalently, \(B \subseteq_a B_{p_i}\) for each \(i\). We shall denote this bornology by \(\vN(V)\).
\end{example}

\begin{example}
    Let \(V\) be a locally convex topological vector space. A subset \(S \subseteq V\) is called \emph{precompact} if for every neighbourhood \(U\) of the origin, there is a finite set \(F\) such that \(S \subseteq F + U\). These precompact sets form a bornology on \(V\), which we denote by \(\Cpt(V)\). To see why any such precompact set is contained in a precompact disk, it suffices to show that the disked hull of a precompact set is precompact. Clearly \(\Cpt(V)\) is closed under scalar multiplication and addition. Consequently, \(t S_1 + (1-t)S_1\) is precompact whenever \(S_1\) is, for every \(t\in [0,1]\). So it suffices to show that the circled hull of a precompact set is precompact. Let \(S \subseteq F + U\), where \(U\) is a circled neighbourhood of the origin. Then the circled hull of \(S\) is contained in \(F^\circhull + U\), where \(F^\circhull\) is the circled hull of \(F\). Consequently, it suffices to show that \(F^\circhull\) is precompact. Since precompact subsets are closed under finite unions, we are reduced to the case where \(F\) is a singleton \(\{x\}\). But then \(F^\circhull\) is the image of the unit disk in \(\C\) under the (continuous) linear map \(\C \to V\), \(\lambda \mapsto \lambda x\). The conclusion now follows from the fact that the continuous image of a precompact set under a linear map is precompact. 
\end{example}

\begin{proposition} \label{prop:Cpt-coarser-vN} Let $V$ be a locally convex topological
vector space. The proecompact bornology is coarser than the von Neumann bornology. 
\end{proposition}
\begin{proof} Let $S \subset V$ be a precompact subset and let us see that it is von Neumann bounded.
This amounts to proving that, given a seminorm $p_i \colon V \to \R_{\ge 0}$, 
the subset $S$ is absorbed by the unit ball $B := B_{p_i}$ of $p_i$. Given that $S$ is precompact
and $B$ is a neighbourhood of the origin, we know that there must exist a finite set 
$F = \{x_1, \ldots, x_k\}$ such that 
\[
S \subset B + F = B_{p_1}(x_1,1) \cup \cdots \cup B_{p_1}(x_n, 1).    
\]
It remains to note that, setting $D = \max_{1 \leq k \leq n} p_i(x_k)$, 
\[
    B_{p_1}(x_1,1) \cup \cdots \cup B_{p_1}(x_n, 1)
    \subset (D+1)B. 
\]
\end{proof}

\subsection{Bounded maps}

We now move to the morphism space in the world of bornologies. Unsurprisingly, we call a linear map \(T \colon V \to W\) \textit{bounded} if for every bounded subset \(S \subseteq V\), the image \(T(S)\) is bounded in \(W\). 
We denote the set of bounded linear maps from \(V\) to \(W\) by \(\Hom(V,W)\). A subset $L \subset \Hom(V,W)$
is \emph{equibounded} if for every 
bounded subset \(B \subseteq V\), the set \(L(B) \defeq \setgiven{T(x)}{T \in L, x \in B}\) 
is bounded in \(W\).

\begin{lemma} \label{lem:equi-born}
    Equibounded subsets form a bornology on \(\Hom(V,W)\).
\end{lemma}
\begin{proof}
    Let \(L\) be an equibounded family of maps, 
    and let \(T \in L^\circhull\) be an element of the circled hull of \(L\). 
    Then there is an \(S \in L\) and a \(\lambda \in \C\) with 
    \(\abs{\lambda} \leq 1\) such that \(T = \lambda S\). Consequently, 
    \(\conv(L)(B) \subseteq L(B)\) for all bounded subsets 
    \(B\) in \(V\), so that \(\conv(L)\) is equibounded as well. 
    The same line of reasoning shows that the given collection of 
    equibounded maps is closed under inclusions. Finite unions are trivial to check. 
    It remains to see that the collection of equibounded linear maps is closed under 
    taking disked hulls. Again, let \(L\) be an equibounded set of linear maps. Then its
    disked hull satisfies \(L^\dhull(B) \subseteq L^\dhull(D) \subseteq L(D)\) for any 
    bounded subset \(B\) contained in a bounded disk \(D\).  
\end{proof}

We denote by \(\mathsf{Born}_\C\), \(\mathsf{Born}_\C^s\) and \(\mathsf{CBorn}_\C\) the categories 
of bornological vector spaces, separated bornological vector spaces and complete bornological vector 
spaces with bounded linear maps (endowed with the equibounded bornology), respectively. After we have 
developed some analysis, we will prove the following:

\begin{lemma}
    If \(W\) is separated (respectively, complete), then \(\Hom(V,W)\) with the equibounded bornology is separated (respectively, complete). 
\end{lemma}


The equibounded bornology can be generalized to the context of multilinear maps. Namely, 
given $n \in \N$ and $V_1, \ldots, V_n, W$ bornological vector spaces, we say that a  
family $L$ of multilinear maps $V_1 \times \cdots \times V_n \to W$ is \emph{equibounded} if
for each collection of bounded subsets $B_i \in \mathcal B(V_i)$, $i \in \{1, \ldots, n\}$ the set
\[
  L(B_1 \times \cdots \times B_n) = \{T(B_1 \times \cdots \times B_n) : T \in L\}  
\]
is bounded on $W$. A multilinear map $T \colon V_1 \times \cdots \times V_n \to W$ is \emph{bounded}
if it maps products of bounded subsets to bounded subsets; equivalently $T$ is bounded if $\{T\}$ 
is equibounded. 
An argument along the lines of Lemma \ref{lem:equi-born} shows that
bounded multilinear maps together with subsets of equibounded
multilinear maps form a bornology; we denote this bornological vector space by
$\Hom^{(n)}(V_1 \times \cdots \times V_n; W)$.
We record the following consequence from the definitions for future usage.
\begin{lemma}\label{lem:bounded-coarsefine-mult}
Let $V_1, \ldots, V_n, W$ be bornological vector spaces and let
$T \colon V_1 \times \cdots \times V_n \to W$ be a bounded multilinear map.
Given bornologies $B_i \subset \mathcal B(V_i)$ 
for each $i \in \{1, \ldots,n\}$ and $B \supset \mathcal B(W)$, the map $T$ 
remains bounded as a multilinear map $(V_1, B_1) \times \cdots \times (V_n, B_n) \to (W,B)$.  
\qed
\end{lemma}

When the bornological vector spaces considered arise from 
locally convex topological vector spaces, 
a natural question to ask is 
what precisely is the relation between continuity and boundedness 
of a multilinear function. Not only that, one may wonder if there are differences
in the notions of boundedness 
when using the von Neumann bornology or the precompact bornology. 
As it turns out, if $V_1, \ldots, V_n$
are additionally Frechét speces then all of 
these notions are one and the same:

\begin{proposition}\label{prop:mult-frechet} Let $V_1, \ldots, V_n$ be Frechét spaces and let $W$ be 
a locally convex topological vector space.  Given an $n$-multilinear
map $T \colon V_1 \times \cdots \times V_n \to W$, 
the following statements are equivalent:
\begin{itemize}
    \item[(i)] The map $f$ is continuous.
    \item[(ii)] The map $f$ is separately continuous, that is, for each $i \in \{1, \ldots, n\}$
    and $v \in \prod_{j \neq i} V_j$, the map $f(v_1, \ldots, v_{i-1}, -, v_{i+1}, \ldots, v_n) \colon V_i \to W$ 
    is continuous. 
    \item[(iii)] Given sequences $(v^1_k)_{k \in \N}, \ldots, (v^n_k)_{k \in \N}$ converging to zero, 
    the sequence $(T(v^1_k, \ldots, v^n_k))_{k \in \N}$ is (von Neumann) bounded in $W$. 
    \item[(iv)] The map $f$ is bounded as a multilinear map $\Cpt(V_1) \times \cdots \times \Cpt(V_n) \to \Cpt(W)$.
    \item[(v)] The map $f$ is bounded as a multilinear map $\vN(V_1) \times \cdots \times \vN(V_n) \to \vN(W)$.
    \item[(vi)] The map $f$ is bounded as a multilinear map $\Cpt(V_1) \times \cdots \times \Cpt(V_n) \to \vN(W)$.
\end{itemize}
\end{proposition}
\begin{proof} It is a classical results that
conditions (i), (ii), and (iii) are equivalent to each other (see 
for instance \cite{??}*{??}). 
Conditions (iv) and (v) both imply (vi) by Lemma \ref{lem:bounded-coarsefine-mult}. 
To see that (vi) implies (iii) it suffices to show that a sequence $S = (v^i_k)_{k\in\N} 
\subset V_i$ converging to zero is a proecompact subset. Indeed, if 
$U$ is a neighbourhood of the origin, we know that there exists $n_0 \in \N$
such that $v^i_k \in U$ if $k \ge n_0$ and so 
\[
S \subset \{v^i_1, \ldots, v^i_{n_0}\} \cup U \subset \{0,v^i_1, \ldots, v^i_{n_0}\} + U.
\]
To conclude we note that (i) implies both (iv) and (v). Both assertions 
stem from the fact that products of precompact (resp. von Neumann bounded) subsets
are precompact (resp. von Neumann bounded) in $V_1 \times \cdots \times V_n$, and 
a continuous map $V_1 \times \cdots \times V_n \to W$ is bounded 
when equipping $V_1 \times \cdots \times V_n$ and $W$ with the precompact (resp. von Neumann)
bornology. 
\end{proof}


\subsection{Convergence}

We continue expanding our dictionaly of notions in the
setting of bornological vector spaces, comparing these 
definititons
with their classical counterparts in functional analysis.

Let $V$ be a bornological vector space. We say that 
a sequence $(x_n)_{n \in \N} \subset V$ is 
\emph{bornologically convergent} to $x \in V$ if there 
exists a bounded disk $D$ such that $(x_n)_{n \in \N}$ converges
to $x$ in $V_D$. Similarly, we say that $(x_n)_{n \in \N}$ 
is a bornological a Cauchy sequence if it 
is a Cauchy sequence in $V_D$ for some bounded disk $D$.

\begin{example} \label{ex:conv-semi}
If $V$ is a semi-normed space and we equip it with its
von Neumann bornology, the usual notions of convergence
and Cauchy sequences coincide with the bornological ones. 
\end{example}

\begin{lemma} \label{lem:finer-conv} Let $V$ a vector space and $B \subset B'$
two bornologies on $V$. If $(x_n)_{n \in \N} \subset V$ is 
bornologically convergent to $x \in V$ for $B$, then it 
is also bornologically convergent to $x \in V$ for $B'$.
\end{lemma}
\begin{proof} By hypothesis $x_n \to x$ in $V_D$
for some disk $D \in B$. This disk is also bounded for $B'$, 
from which the lemma follows.
\end{proof}

\begin{proposition}[Uniqueness of limits] A 
bornonological vector space $V$ is separated 
if and only if every sequence converges 
to at most one point. 
\end{proposition}
\begin{proof} Suppose that a sequence $(x_n)_{n \in \N}$ 
converges simlultaneously to two
elements $x,y \in V$. By definition, there exist 
bounded disks $D, D' \in \mathcal B_d(V)$ such that 
$x_n \to x$ in $V_D$ and $x_n \to y$ in $V_{D'}$. 
Given that $V$ is separated, there exists a 
norming disk $D''$ containing both $D$ and $D'$. 
Note that in particular we have continuous
inclusions $(V_D, p_D) \to (V_{D''}, p_{D''})$
and $(V_{D'}, p_{D'}) \to (V_{D''}, p_{D''})$.
It follows that $(x_n)_{n \in \N}$ converges to both 
$x$ and $y$ in $V_{D''}$. 
Since $V_{D''}$ is a normed space, it is Hausdorff; 
this implies that $x = y$ as wanted.

Conversely, suppose that convergent 
sequences in $V$ have unique limits. 
Given a bounded subset $B$, we will see that its
disked hull $D$ is normed. If it were not the case,
then there must exist $z \in V_D \setminus \{0\}$ such that 
the gauge semi-norm $p_D$
associated to $D$ satisfies 
$p_D(z) = 0$. This is absurd, as it would 
imply that the constant sequence $z_n := z$
converges both to $z$ and $0$. 
\end{proof}

Example \ref{ex:conv-semi} can be 
generalized to Frechét spaces, as the following result shows.

\begin{theorem} \label{thm:frechet-convergence}
Let $V$ be a Frechét space and $(x_n)_{n \in \N} \subset V$.
The following statements are equivalent:
\begin{itemize}
    \item[(i)] The sequence $(x_n)_{n \in \N}$ is convergent (resp. Cauchy).
    \item[(ii)] The sequence $(x_n)_{n \in \N}$ is bornologically convergent (resp. Cauchy)
    for the von Neumann bornology.
    \item[(iii)] The sequence $(x_n)_{n \in \N}$ is bornologically convergent (resp. Cauchy)
    for the precompact bornology.
\end{itemize}
\end{theorem}
\begin{proof} 
The fact that (ii) and (iii)
are equivalent statements
is left as Exercise \ref{ex:conv-Cpt=vN}; we prove that (i) 
is equivalent to (ii).

Suppose that $(x_n)_{n\in\N}$ 
is bornologically convergent for the von Neumann bornology to a point $x \in V$ and
let $\mathcal F = \{p_i\}_{i \in I}$ be a family of semi-norms defining the topology of $V$.
By definition, there exists a von Neumann bounded disk $D$ such that
$x_n \to x$ in $V_D$. Since $D$ is bounded 
the unit ball $B_i$ of each semi-norm $p_i$ 
absorbs $D$, which implies that $p_i(x_n) \to p(x)$ for all semi-norms $p_i \in \mathcal F$. 
Because the topology on $V$ is initial 
with respect to the family $\mathcal F$, 
it follows that $(x_n)_{n \in \N}$ 
converges to $x$ in $V$. The statement 
for Cauchy sequences is proved analogously.

Conversely, suppose that $X = (x_n)_{n \in \N}$ converges topologically to a point $x \in V$.
Without loss of generality we may suppose that the family of semi-norms 
defining the is increasing.
In particular, we may fix a countable set $(V_n)_{n\in\N}$ of 
circled, convex, decreasing set of neighbourhoods of $0$.
By hypothesis, for each $n \in \N$ all but finitely many elements of $X$
lie in $V_n$. Thus, upon dilating $V_n$ by a suitable positive constant $\lambda_n$,
we have that $X \subset \lambda_n V_n$.

Put $B = \cap_{n \ge 1} n\lambda_n V_n$. It follows that $B$ is a bounded disk; we shall presently 
see that $x_n \to x$ in $V_B$. It suffices to see that given $M \ge 1$,
all but finitely many elements of $X$ lie in $(1/M)B$. Note that
\[
(1/M)B
= \bigcap_{n \le M} (n/M) \lambda_n V_n 
\cap \bigcap_{n > M} (n/M) \lambda_n V_n.
\]
Since $X \subset \lambda_n V_n \subset (n/M)  \lambda_n V_n$ for all $n > M$, it remains 
to show that all but finitely many elements of $X$ lie in $\bigcap_{n \le M} n\lambda_n V_n$, which follows from the fact that this is a neighbourhood of
the origin and $x_n\to 0$ topologically.
The same argument replacing $X$ by
$Y := \{x_n-x_m : n,m \in\N\}$ yields the result for Cauchy sequences.
\end{proof}


%\begin{definition}[absolutely summable, continuous and smooth functions] Let $V$ be a bornological vector space and $X$ a set.
%A function $f \colon X \to V$ is said to be \emph{absolutely summable} if $p_D \circ f \colon X \to \R_{\ge 0}$ 
%is absolutely summable for some bounded disk $D \subset V$.
%The space of such functions will be denoted $\ell^1(X,V)$.

%Suppose additionally that $X$ is a compact
%topological space. A function $f \colon X \to V$ 
%is said to be
%continuous if there exists $D\in %\mathcal B_d(V)$
%such that it can be corestricted to a 
%continuous function $X \to V_D$. 
%The space of continuous will be denoted $\mathcal C(X,V)$. 

%A subset $S \subset \mathcal C(X,V)$ is \emph{uniformly bounded} if they the elements of $S$ can all be corestricted to $V_D$
%for some bounded disk $D$. We say that $S$ is \emph{uniformly
%continuous} if it is uniformly bounded in the sense above
%and additionally the family $\{f|^{V_D} : f \in S\}$ is uniformly continuous.
%\end{definition}

%\begin{proposition} If 
%$V$ is a Frechét space, 
%a series is absolutely summable
%if it is so for the von Neumann
%or precompact bornologies.
%\end{proposition}
%\begin{proof}
%\end{proof}

%The collections of uniformly bounded and 
%uniformly continuous maps endow 
%$\mathcal C(X,V)$ with
%two bornological vector space structures. 
%Unless explictly mentioned, we consider $\mathcal C(X,V)$
%with the uniformly continuous bornology. 

%\begin{example} If $X$ is a 
%compact topological space, 
%then $C(X, \C) = C(X)$ is a Banach
%space. By the Arzelà-Ascoli
%theorem, its bornology of 
%uniformly continuous subsets 
%coincides with the precompact bornology. In contrast, its
%bornology of uniformly bounded
%functions 
%is the von Neumann bornology.
%\end{example}

%\begin{theorem}[\cite{meyer-metrizable}*{Theorem 3.7}] Let $V$ be a Frechét
%space and $X$ a compact topological
%space. There is a bornological isomorphism
%\[
%\mathcal C(X, \Cpt(V)) \simeq \Cpt(\mathcal C(X,V)).
%\]
%Likewise, a subset of $\mathcal %C(X,V)$ is von Neumann bounded %if 
%and only if it is uniformly %bounded
%as a subset of $\mathcal C(X, %\vN(V))$.
%\qed
%\end{theorem}

\subsection{Bornological closure}

\subsection{The precompact bornology}

\subsection{The category of bornological vector spaces}



In this section, we will study categorical operations in the category of bornological vector spaces \(\mathsf{Born}_\C\) and its relatives, namely separated \(\mathsf{Born}_{\C}^s\) and complete bornological vector spaces \(\mathsf{CBorn}_\C\).


The first operations we look at are \textit{subspaces} and \textit{quotients}. Let \((V, \mathcal{B}(V))\) be a bornological vector space and \(W \subseteq V\) a vector subspace. 

\begin{itemize}
    \item The \textit{subspace bornology} on \(W\) is the bornology whose bounded subsets are those which are bounded in \(V\). That is, 
    \[\mathcal{B}(W) = \setgiven{B \subseteq W}{B \in \mathcal{B}(V)} = \setgiven{B \cap W}{B \in \mathcal{B}(V)}.\]
    \item The \textit{quotient bornology} on \(V/W\) is the bornology \[\mathcal{B}(V/W) \defeq \setgiven{p(B)}{B \in \mathcal{B}(V)},\] where \(p \colon V \to V/W\) is the quotient map.
\end{itemize}

We always equip subspaces and quotients with the subspace and quotient bornologies. These satisfy the property that a linear map \(f \colon X \to W\) is bounded in the subspace bornology on \(W\) if and only if it is bounded in \(V\), and a map \(g \colon V/W \to X\) is bounded if and only if \(g \circ p \colon V \to X\) is bounded. Here \(X\) is an arbitrary bornological vector space. In other words, the subspace and quotient bornologies play the role of the subspace and quotient topologies in the theory of topological vector spaces.



\begin{lemma}[Sequential criterion for separatedness]\label{lem:sequential-separatedness}
    Let \(V\) be a bornological vector space. Then \(V\) is separated if and only if every convergent sequence has a unique limit. 
\end{lemma}

\begin{proof}
    \textbf{Redundancy. Perhaps just point out in the previous time this was proven that separatedness means \(\overline{\{0\}} = \{0\}\).} Let \((x_n)\) be a convergent sequence in \(V\), and let \(V\) be separated. Suppose this sequence has two limits \(x\) and \(y\), then the sequence \(0 = x_n - x_n\) has limit \(x - y\). So it suffices to show that the constant null sequence has limit zero, or equivalently, \(\overline{\{0\}} = \{0\}\). Suppose \(z\) is a limit of \(0\). By definition of bornological convergence this means that there is a null sequence of positive real numbers \((\epsilon)_n\) such that \(z \in \epsilon_n D\), for a bounded disk \(D\) and each \(n\). But then \(\varrho(z) = 0\) and since \(\varrho\) is a norm by separatedness, we are done.

    Conversely, suppose \(\overline{\{0\}} = \{0\}\). Let us hypothetically suppose that for a bounded disk \(D\), the gauge semi-norm is not a norm. Then there is a \(z \neq 0\) such that \(\varrho_D(z) = 0\), where \(D\) is a bounded disk. It is then left as an exercise to check that there is a sequence \((\lambda_n)\) converging to zero such that \(z \in \lambda_n D\). This implies that the constant \(0\) sequence bornologically converges to \(z\), which is a contradiction. 
\end{proof}

\begin{lemma}\label{lem:extensions-inherentance}
Let \(V\) be a bornological vector space and \(W\) a subspace. Equip \(W\) and \(V/W\) with the subspace and quotient bornologies. We then have the following:
\begin{enumerate}
    \item\label{eq:1} \(W\) is separated if \(V\) is separated;
    \item\label{eq:2} \(V/W\) is separated if and only if \(W\) is closed;
    \item\label{eq:3} Suppose \(V\) is complete. Then \(W\) is complete if and only if \(W\) is closed, in which case \(V/W\) is complete.
\end{enumerate}
\end{lemma}

\begin{proof}
Part \ref{eq:1} is clear. Now suppose \(V/W\) is separated. Then \(\{0\}\) is closed in \(V/W\) by the proof of Lemma \ref{lem:sequential-separatedness}. Since a bounded linear map is continuous for the bornological topology of bornologically closed subsets, \(W = p^{-1}(\{0\}) \subseteq V\) is closed in \(V\). Conversely, if \(V/W\) is not separated, then there is \(0 \neq x \in V/W\) and a null sequence \((\lambda_n)\) such that \(x \in \lambda_n D\) for some bounded disk \(D \subseteq V/W\). Now there are \(y \in V\) and \(S \in \mathcal{B}(V)\) such that \(g(y) = x\) and \(p(S) = D\). Writing \(x = \lambda_n x_n\), we can find \(y_n \in S\) such that \(p(y_n) = x_n\). Consequently, \(p(y - \lambda_n y_n) = 0\), so that \(y - \lambda_n y_n \in W\). It converges to \(y\), which does not lie in \(W\) since \(p(y) = x \neq 0\) by hypothesis. So what we have found is a convergent sequence in \(W\) whose limit is not in \(W\), so \(W\) cannot be closed. This proves \ref{eq:2}. Finally,  suppose \(W\) is complete. Then it is closed since for every bounded disk \(D\), \(W_D\) is a Banach space for the gauge norm on \(D\), which implies in particular that the disk \(D\) is norming. The converse is left as an exercise. 
\end{proof}


We now revisit the relations we had encountered \textbf{some time ago (draw that commuting diagram)}. We had canonical inclusion functors \[\mathsf{CBorn}_{\C} \hookrightarrow \mathsf{Born}_{\C}^s \hookrightarrow \mathsf{Born}_\C\] between the categories of complete, separated and (all) bornological vector spaces. We now construct canonical maps going in the other direction.

\begin{definition}\label{def:completion-separated}
Let \(V\) be a bornological vector space. 
\begin{itemize}
    \item The \textit{separated quotient} of \(V\) is the vector space \(\mathsf{sep}(V) \defeq V/\overline{\{0\}}\) with the quotient bornology. It has the universal property that for any separated bornological vector space \(W\), there are natural isomorphisms \(\mathsf{Hom}(V,W) \cong \mathsf{Hom}(\mathsf{sep}(V), W)\).
    \item The \textit{completion} of a separated bornological vector space is a complete bornological vector space \(\comb{V}\)  characterised by the universal property \(\mathsf{Hom}(V,W) \cong \mathsf{Hom}(\comb{V},W)\).
\end{itemize}
\end{definition}

Lemma \ref{lem:sequential-separatedness} shows that the separated quotient is indeed separated. Note that although the universal property of completions determines the object \(\comb{V} \in \mathsf{CBorn}_\C\), we have not yet shown its existence. We shall do this in a later section. 

We now look at \textit{kernels and cokernels}. Let \(f \colon V \to W\) be a bounded linear map between bornological vector spaces. 

\begin{itemize}
    \item The \textit{kernel} of \(f\) is the vector subspace \(\ker(f) = \setgiven{x \in V}{f(x) = 0}\) with the subspace bornology it inherits from \(V\).
    \item The \textit{cokernel} of \(f\) is a vector space \(\coker(f) = V/f(W)\) with the quotient bornology it inherits from \(V/W\).  
\end{itemize}

Just as in the vector space case, one checks that \(\ker(f)\) and \(\coker(f)\) are really kernels and cokernels in the categorical sense. If the reader is unfamiliar with category theory, this is an instructive exercise. 

Now consider a bounded linear map \(f \colon V \to W\) between separated bornological vector spaces. Then by Lemma \ref{lem:extensions-inherentance} applied to \(V\), we see that \(\ker(f)\) is separated. It also satisfies the universal property of kernels since this can already be checked in \(\mathsf{Born}_\C\). Finally, suppose \(V\) is complete. Then since \(V/\ker(f) \subseteq W\) is separated (because \(W\) is separated), \(\ker(f)\) is closed by Lemma \ref{lem:extensions-inherentance}. Another application of Lemma \ref{lem:extensions-inherentance} shows that \(\ker(f)\) is complete, being a closed subspace of \(V\).  Note that so far there has been no difference between ordinary vector spaces and topological or bornological vector spaces. That is, kernels are just vector space kernels with a canonical bornology appended to it.


What happens to cokernels? This is somewhat tricky and should be earmarked as the starting point of the deviation from linear algebra. Since the image of a bounded linear map need not be closed, the quotient \(W/f(V)\) need not be separated in light of Lemma \ref{lem:extensions-inherentance}. So to make it separated, we need to close the image of \(f\), and by Lemma \ref{lem:extensions-inherentance}, \(W/\overline{f(V)}\) is indeed separated (and complete, if \(W\) is complete). It is now an exercise to check that the quotient \(W/\overline{f(V)}\) satisfies the universal property of cokernels in both \(\mathsf{Born}_{\C}^s\) and \(\mathsf{CBorn}_{\C}\). 

The conclusion is therefore that the categories \(\mathsf{Born}_\C\), \(\mathsf{Born}_{\C}^s\), and \(\mathsf{CBorn}_\C\) have kernels and cokernels. 


We now come to direct sums and direct products. Let \((V_i)\) be a family of bornological vector spaces, where \(i\) is an arbitrary indexing set. We define the direct sum \(\bigoplus_{i \in I} V_i\) as the ordinary vector space direct sum with the bornology generated by \(\iota_i(S_i) \subseteq  \bigoplus_{i \in I} V_i\), where \(S_i\) is bounded in \(V_i\) for each \(i \in I\). By construction this says that the inclusion maps \(\iota_i \colon V_i \to \bigoplus_{i \in I} V_i\) are bounded for each \(i \in I\). Dually, the direct product of the given family is defined as the usual product \(\prod_{i \in I} V_i\) of vector spaces, together with the bornology where a subset \(S \subseteq \prod_{i \in I} V_i\) is bounded if and only if the projections \(p_i(S) \subseteq V_i\) is bounded for each \(i\). One checks just as in the case of vector spaces that the direct sum and product with their respective bornologies satisfy the universal property of direct sums and products. With this, we conclude the following:


\begin{theorem}\label{thm:bornologies-complete}
The categories \(\mathsf{Born}_{\C}\), \(\mathsf{Born}_{\C}^s\) and \(\mathsf{CBorn}_\C\) are both complete and cocomplete. That is, they have all limits and colimits. 
\end{theorem}

\begin{proof} We have seen that any of the categories considered have both 
products and coproducts. Moreover, since these are additive categories which 
have kernels and cokernels, they have equalizers and coequalizers. 
The result now follows from the fact that any category with products (resp. coproducts) and equalizers (resp. coequalizers) is complete (resp. cocomplete); see for example
\cite{riehl}*{Theorem 3.4.12}. 
\end{proof}




\subsection{The projective bornological tensor product}

In this section, we introduce the \textit{bornological tensor product}. Let \(V\) and \(W\) be bornological vector spaces. We define the \textit{(projective) tensor product bornology} as the bornology generated by subsets \(B_1 \otimes B_2\), where \(B_1\) and \(B_2\) are bounded disks in \(V\) and \(W\). In other words, a subset is bounded if and only if it is contained in the disked hull of \(B_1 \otimes B_2\) for \(B_1 \subseteq V\) and \(B_2 \subseteq W\) bounded disks. The bornological tensor product satisfies the universal property that for any bounded bilinear map \(f \colon V \times W \to Z\) into a bornological vector space, there is a unique bounded linear map \(\tilde{f} \colon V \otimes W \to Z\) such that \(\tilde{f} \circ b = f\), where \(b \colon V \times W \to V \otimes W\) is the canonical map.

Now let \(V\) and \(W\) be complete, bornological vector spaces. Their \textit{completed bornological tensor product} \(V \haotimes W\) is defined as the (yet to be defined) bornological completion \[V \haotimes W = \comb{(V \otimes W)}\] of the tensor product \(V \otimes W\) with respect to the tensor product bornology. 

We now compare the bornological tensor product with the topological tensor product, or more precisely, the projective topological tensor product. Recall that the completed projective topological tensor product \(V \haotimes_{\pi} W\) of two  complete, locally convex topological vector spaces is defined as complete locally convex topological vector space representing jointly continuous bilinear maps \(V \times W \to X\) into a complete, locally convex topological vector space. In this direction, we have the following:


\begin{theorem}\label{thm:Grothendieck-tensor}
Let \(V\) and \(W\) be Frech\'et spaces. The canonical continuous bilinear map \(b \colon V \times W \to V \haotimes_{\pi} W\) induces a bornological isomorphism \(\mathsf{Cpt}(V) \haotimes \mathsf{Cpt}(W) \cong \mathsf{Cpt}(V \haotimes_{\pi} W)\). 
\end{theorem}

To prove this, we will use the following theorem (Grothendieck):
\begin{theorem}\label{thm:Grothendieck-x-as-a-serie}
Let \(V\) and \(W\) be Frechét spaces. Any \(x \in V \haotimes_{\pi} W\) is of the form \[x = \sum_{n \in \N} \lambda(n) y_n \otimes  z_n \] with null sequences \((y_n) \in V\), \((z_n) \in W\) and \(\lambda \in l^1(\N)\). 

Furthermore, if \((x_k)_{k\in \N}\) is a null sequence in \(V \haotimes_{\pi} W\), then there are null sequences  \((y_n)\) in \(V\) , \((z_n) \) in \(W\) ; and a null sequence \(\lambda_k \in l^1(\N)\) with \[x_k = \sum_{n \in \N} \lambda_k (n) y_n \otimes  z_n \]
\end{theorem}

A proof could be found in \cite{grothendieck}*{pages 51 and 57}.

\begin{proof}
Let \(X\) be a complete bornological vector space. We want to show that a bounded bilinear map \(\mathsf{Cpt}(V) \times \mathsf{Cpt}(W) \to X\) extends uniquely to a bounded linear map \(\mathsf{Cpt}(V \haotimes_{\pi} W)\). Since \(\mathsf{Cpt}(V) \haotimes \mathsf{Cpt}(W)\) is universal for this property, the result will then follow from the Yoneda Lemma. 

Let \(f\colon \mathsf{Cpt}(V) \times \mathsf{Cpt}(W) \to X\) be a bounded bilinear map. We want to show that there is a unique bounded linear map \(\tilde{f}\colon \mathsf{Cpt}(V \haotimes_{\pi} W) \to X\) such that \(\tilde{f} \circ b = f\). Suppose such a map exists, then 
\[\tilde{f}(x) = \sum_{n \in \N} \lambda_n \tilde{f}(y_n \otimes  z_n) = \sum_{n \in \N} \lambda_n f(y_n, z_n),\] where we have used that any element \(x \in V \haotimes_{\dvr} W\) can be written as a series \(x = \sum_{n \in \N} \lambda_n y_n \otimes z_n\), where \(\lambda \in l^1(\N)\), \((y_n) \in V\) and \((z_n) \in W\) are null sequences as stated by the theorem \ref{thm:Grothendieck-x-as-a-serie}. As a consequence, if such an \(\tilde{f}\) exists, it is completely determined by \(f\) and hence unique. 

Now define \(\tilde{f} \colon V \haotimes_{\pi} W \to X\) as \[\tilde{f}(x) = \sum_{n \in \N} \lambda_n f(y_n,z_n),\] where \(x = \sum_{n \in \N} \lambda_n y_n \otimes z_n,\) for null sequences \((y_n) \in V\), \((z_n) \in W\) and \(\lambda \in l^1(\N)\). We need to show that this formula is independent of the choices of the representing sequences \((y_n)\) and \((z_n)\). It suffices to show that for \(0 = \sum_{n \in \N} \lambda_n y_n \otimes z_n\), we have \(\tilde{f}(0) = 0\). Let \(x_k \defeq \sum_{n \leq k} \lambda_n y_n \otimes z_n\).  Then by hypothesis, \(\lim_{k \to \infty} x_k = 0\), so that \(x_k\) is a null-sequence in \(V \haotimes_{\pi} W\). By theorem  \ref{thm:Grothendieck-x-as-a-serie}, there are null sequences \(\lambda_{k}' \in l^1(\N)\) and \((y_n')\) and \((z_n')\), such that \(\sum_{n \leq k} \lambda_n y_n \otimes z_n = x_k = \sum_{n \in \N}\lambda_{k,n}' y_n' \otimes z_n' \in V \haotimes_{\pi} W\). Since \(f\) is a bounded bilinear map, the subset \(\setgiven{f(y_n',z_n')}{n \in \N}\) is bounded in \(X\). Consequently, \[\lim_{k \to \infty} \sum_{n \in \N} \lambda_{k,n}' f(y_n',z_n') = 0.\] The term inside the limit above is \(\tilde{f}\) applied to \(x_k\). But applying \(\tilde{f}\) to \(x_k\) is also \(\sum_{n \leq k} \lambda_n f(y_n, z_n)\), whose limit as \(k \to \infty\) is \(\sum_{n \in \N} \lambda_n f(y_n, z_n)\). Summarily, \[\tilde{f}(0) = \sum_{n \in \N} \lambda_n f(y_n,z_n) = \lim_{k \to \infty} \sum_{n \in \N} \lambda_{k,n}'f(y_n', z_n') = 0,\] as required. 

Finally, it remains to show that \(f \colon \mathsf{Cpt}(V \haotimes_\pi W) \to X\) is bounded. Suppose \(S \subseteq V \haotimes_{\pi} W\) is precompact. Then by \textbf{cite Grothendieck}, \(S\) is contained in the complete disked hull of a null sequence \((x_n)\). \textbf{Note that we don't really need the explicit description of the complete disked hull here.} Consequently, the complete disked hull of \(\setgiven{\tilde{f}(x_n)}{n \in \N}\) is bounded in \(X\), as required. 
\end{proof}


What we have shown therefore is that there is a well-behaved notion of a tensor product in the bornological framework, which generalises the completed projective topological tensor product for Frech\'et spaces. But we get more in this framework, namely, 

\begin{theorem} \label{thm:tensor-hom}
    For (separated) bornological vector spaces \(V\) and \(W\), we have \(\Hom(V, \underline{\mathsf{Hom}}(W,Z)) \cong \Hom(V \otimes W, Z)\), where \(\otimes\) denotes the bornological tensor product. For complete bornological vector spaces \(V\) and \(W\), we have \(\Hom(V, \underline{\mathsf{Hom}}(W,Z)) \cong \Hom(V \haotimes W, Z),\), where \(\haotimes\) is the completed projective tensor product. In other words, we have a tensor-Hom adjunction. 
\end{theorem}
\begin{proof} Left as Exercise \ref{ex:tensor-hom}. 
\end{proof}


\subsection{From bornologies to inductive systems}

In this section, we define a category that is closely related to the category of bornological vector spaces, namely, the category of inductive systems over semi-normed vector spaces. But before we get there, we recall some generalities on inductive systems. Throughout this section, let \(\mathcal{C}\) be an additive category with finite limits. This means that \(\mathcal{C}\) has kernels, cokernels, finite products and finite coproducts. 

An \textit{inductive system} over \(\mathcal{C}\) is a functor \(F \colon I \to \mathcal{C}\), where \(I\) is a directed set. Concretely, this corresponds to objects \((F_i)_{i \in I}\) in \(\mathcal{C}\), and for \(i \leq j\), a morphism \(F_i \to F_j\), which we call structure maps of the inductive system. We often denote inductive systems by \((A_i)_{i \in I}\). The inductive systems over \(\mathcal{C}\) assemble into a category where the morphisms are defined as \[\Hom(A,B) = \varprojlim_{i \in I} \varinjlim_{j \in J} \Hom(A_i,B_j),\] where \(A = (A_i)_{i \in I}\) and \(B = (B_j)_{j \in J}\). More concretely, we can represent any morphism \(f \colon A \to B\) of inductive systems by a family of morphisms \(f_i \colon A_i \to B_{j(i)}\) for \(j \colon I \to J\) such that for all \(k \geq i\), there is an \(l \in J\) with \(l \geq j(i), j(k)\) for which the  diagram  
\[
\begin{tikzcd}
A_i \arrow{r}{f_i} \arrow{d}{} & B_{j(i)} \arrow{dr}{} & \\
A_k \arrow{r}{f_k} & B_{j(k)} \arrow{r}{} & B_l
\end{tikzcd}
\] commutes, where the unlabelled arrows are the structure maps of the inductive system. Actually, it can be arranged that a morphism \(f\) be represented as a family \((f_l \colon A_l \to B_l)_{l \in L},\) by appropriately modifying the indexing directed set of \(A\) and \(B\). 

The following is relevant for our theory:

\begin{theorem}
The categories \(\mathsf{Ind}(\mathsf{Norm}_\C^{1/2})\), \(\mathsf{Ind}(\mathsf{Norm}_\C)\) and \(\mathsf{Ind}(\mathsf{Ban})\) are bicomplete.
\end{theorem}

\begin{proof}
    Since we are working additive categories, it suffices to show that each of these categories has kernels, cokernels, coproducts and products. We also only discuss the proof for \(\mathsf{Ind}(\mathsf{Norm}_\C^{1/2})\), as the proof for the other two categories works the same way.

We first discuss kernels. Let \(f \colon A \to B\) be a morphism in \(\mathsf{Ind}(\mathsf{Norm}_\C^{1/2})\). As already remarked, we can represent \(f\) by a diagram \((f_i \colon A_i \to B_i)\) of morphisms in \(\mathcal{C}\), after possibly changing the indexing set. Now by hypothesis, \(\ker(f_i)\) exists in \(\mathcal{C}\). We leave it for the reader to check the universal property, namely, morphism an equaliser diagram \(Z \to A \overset{f}\rightrightarrows B\) factors uniquely through \((\ker(f_i))_{i \in I}\). This shows that \(\ker(f) \cong (\ker(f_i))_{i \in I}\). The dual argument applies to \(\coker(f) \cong (\coker(f_i))_{i \in I}\).  

For coproducts, let \((A^{(k)})_{k \in K}\) be a collection of inductive systems, where \(A^{(k)} = (A_i^{(k)}, \alpha_i^{j,(k)})_{i \in I_k}\). Define \((F,\varphi)\), where \(F \subseteq K\) is finite and \(\varphi \colon F \to I\) assigns to each \(k \in F\), and element of \(I_k\). We define a partial order on this collection \(\Lambda\) as follows: \[(F,\varphi) \leq (F',\varphi') \text{ if and only if } F \subseteq F' \text{ and } \varphi(k) \leq \varphi'(k) \text{ for } k \in F.\] We leave it for the reader to check that this really is a directed set. With this as the indexing category, we can now define the inductive system \[\bigoplus_{k \in K} A^{(k)} \colon (F,\varphi) \mapsto A_{(F, \varphi)} = \bigoplus_{k \in F}A_{\varphi(k)}^{(k)}.\] There are canonical maps \(A^{(k)} \to \bigoplus_{k \in K}A^{(k)},\) induced by \(A_i^{(k)} \to A_{(\{k\}, k \mapsto i)} = A_i^{(k)}.\) The universal property follows from the fact that the underlying category has finite coproducts.

For products, consider the set \(\Phi \defeq \setgiven{(\varphi_1,\varphi_2)}{\varphi_1 \colon K \to I, \varphi_2 \colon K \to \N_{\geq 1}, \varphi_1(k) \in I_k}\) with the relation 
\[(\varphi_1, \varphi_2) \leq (\psi_1, \psi_2) \text{if and only if } \varphi_1(k) \leq \psi_1(k), \varphi_2(k) \leq \psi_2(k).\] This is directed set. Now for \(\varphi = (\varphi_1, \varphi_2) \in \Phi,\) define \[A_\varphi \defeq \setgiven{(x_k) \in \prod_{k \in K} A_{\varphi_1(k)}^{(k)}}{\frac{\norm{x_k}}{\varphi_2(k)} \text{ is bounded}}.\] With the semi-norm defined by \(\varrho((x_k)) = \sup_{k} \frac{\norm{x_k}}{\varphi_2(k)},\) we get an inductive system of semi-normed spaces. Finally, the projections defined by \(A_{\varphi} \to A_{\varphi(k)}^{(k)}\) assemble to yield morphisms of inductive systems \((A_{\varphi})_{\varphi \in \Phi} \to A^{(k)}\) for each \(k\). These satisfy the universal property of products, which is left for the reader to check.  
\end{proof}


We now relate the categories of bornological vector spaces with the categories of inductive systems of topological vector spaces. The bornological vector spaces we have built have essentially been constructed out of topological vector spaces associated to unit balls or bounded disks. In this section, we make this construction precise. Let \(V\) be a bornological vector space. The collection of its bounded subsets \(\mathcal{B}(V)\) and bounded disks \(\mathcal{B}_d(V)\) are directed sets with respect to inclusion and absorption. As the latter family is cofinal in the former, we restrict attention to \(\mathcal{B}_d(V)\) with respect to absorption. 

Now each element \(D\) of \(\mathcal{B}_d(V)\) is the unit ball of a semi-normed space \(V_D = \R_{\geq 0} \cdot D\). Suppose we take \(D \subseteq_a D'\), we get an bounded linear inclusion \(V_D \subseteq V_D'\). It is easy to see that the assignment \(\mathcal{B}_d(V) \ni D \mapsto V_D \in \mathsf{Norm}^{1/2}\) is a functor. That is, \((V_D)_{D \in \mathcal{B}_d(V)}\) is a directed set. Furthermore, if \(f \colon V \to W\) is a bounded linear map, then for any bounded disk \(S\subseteq V\), there is a bounded disk \(T\) containing \(f(S)\), so we get an induced linear map \(V_S \to W_T\). These actually combine to a morphism of inductive systems \((V_S)_{S \in \mathcal{B}_d(V)} \to (W_T)_{T \in \mathcal{B}_d(W)}\) \textbf{check this}. In summary, we obtain a functor 

\[\mathsf{diss} \colon \mathsf{Born}_\C \to \mathsf{Ind}(\mathsf{Norm}_{\C}^{1/2}),\] which we call the \textit{dissection functor}. The same assigment restricts to functors \[\mathsf{diss} \colon \mathsf{Born}_\C^s \to \mathsf{Ind}(\mathsf{Norm}_\C), \quad \mathsf{CBorn}_\C \to \mathsf{Ind}(\mathsf{Ban}_\C),\] since for bounded disks \(D\) in separated (respectively complete) bornological vector spaces, the associated semi-normed vector space \(V_D\) is normed (respectively, Banach), by definition.


There is also an obvious functor from inductive systems of semi-normed spaces to bornological vector spaces, namely, the \textit{inductive limit functor}. Note that the inductive limit exists as the category of bornological vector spaces is bicomplete. We describe it more concrete as follows: given an inductive system \((V_i)_{i \in I}\) of semi-normed spaces, we take its inductive limit \(\varinjlim_{i \in I} V_i\) as an ordinary vector space, and equip it with the bornology where a subset is bounded if and only if it is contained in the image of \(V_i\) for some \(i \in I\). 

Now suppose \((V_i)_{i \in I}\) is an inductive system of normed spaces. Taking the bornological inductive limit as above need not be separated, so we need to take the separated quotient \(\mathsf{sep} \varinjlim (V_i) = (\varinjlim_{i \in I} V_i)/\overline{\{0\}}\) to ensure this. The same construction also yields a complete bornological vector space if we start with an inductive system of Banach spaces. We now analyse the relationship between the dissection and (separated) inductive limit functors:

\begin{proposition} We have the following:
\begin{enumerate}
    \item\label{prop:1-ind} We have natural isomorphisms \(\mathsf{sep} \varinjlim\mathsf{diss} (V) \cong V\) for \(V \in \mathsf{Born}_\C^s\);
    \item\label{prop:2-ind} The functor \(\mathsf{diss} \colon \mathsf{Born}_\C^s \to  \mathsf{Ind}(\mathsf{Norm}_\C)\) is fully faithful;
    \item\label{prop:3-ind} It commutes with inverse limits and direct sums;
    \item\label{prop:4-ind} It commutes with \(\mathsf{Hom}\), or more generally multilinear maps; that is, 
    \[\mathsf{Hom}^{(n)}(\mathsf{diss}(V_1) \times \dotsc \times \mathsf{diss}(V_n), \mathsf{diss}(W)) \cong \mathsf{diss}(\mathsf{Hom}^{(n)}(V_1 \times \dotsc \times V_n, W)).\]
    \item\label{prop:5-ind} \(\mathsf{sep} \varinjlim \colon \mathsf{Ind}(\mathsf{Norm}_\C) \to \mathsf{Born}_\C^s \) is left adjoint to the dissection functor;
    \item\label{prop:6-ind} The assertions above hold for \((\mathsf{Born}_\C, \varinjlim, \mathsf{diss})\) and \((\mathsf{CBorn}_\C, \mathsf{sep}\varinjlim, \mathsf{diss})\). 
\end{enumerate}

\end{proposition}

\begin{proof}

Let \(V\) be a separated bornological vector space. We first show that there is a vector space isomorphism between \(\mathsf{sep} \varinjlim \mathsf{diss}(V)\) and \(V\). To this end, we first write \(V = \bigcup V_D\) for normed spaces \(V_D\). This is precisely the vector space inductive limit \(\varinjlim \mathsf{diss}(V)\). To see that this is also the bornological inductive limit, we see that a subset is bounded if and only if it is contained in \(V_D\) for some bounded disk \(D\). Finally, since \(V\) is separated, taking separated quotients does nothing, proving \ref{prop:1-ind}. 


\end{proof}

\subsection{Quasi-abelian categories}

In this section we fix an additive category  $\mathsf C$ 
with kernels and cokernels. 

Recall that the \emph{image} of a morphism $f \colon X \to Y$
is defined as $\im(f) := \ker(\coker(f))$, and its \emph{coimage}
as $\coim(f) := \coker(\ker(f))$.
There always exists a canonical comparison map between 
the coimage and the image, which fits in the following diagram:
\[
\begin{tikzcd}
\ker(f) \arrow{r} & X \arrow{d} \arrow{r} & Y \arrow{d}\arrow{r} & \coker(f)\\
& \coim(f) \arrow[dashed]{r}[above]{\exists!}& \im(f) & 
\end{tikzcd}
\]
We say that $f$ is \emph{strict} if the dashed arrow above is an isomorphism. Informally, 
a morphism is strict if it 
satisfies Noether's first isomorphism theorem. 

We say that $\mathsf C$ is \emph{abelian}
if all morphisms are strict. These are categories in which classical 
homological algebra takes place; important examples include that 
of the category of modules over a ring and sheaves 
of abelian groups over a space. None of the 
categories that we have considered so far are abelian. To see this, 
we first record the following lemma.

\begin{proposition}\label{prop:ab-epimono=iso} Let $f \colon X  \to Y$ be a strict morphism in $\mathsf C$. 
If $f$ is both a monomorphism and an epimorphism, then 
it is an isomorphism.
\end{proposition}
\begin{proof}
If $f$ is a monomorphism, then $\ker(f) = 0$. Likewise, the fact that $f$
is an epimorphisms implies that $\coker(f) = 0$. It follows that
$\coim(f) = \coker(0 \to X) = X$ and $\im(f) = \ker(Y \to 0)$; consequently, 
the comparison map $\coim(f) \to \im(f)$ can be identified with $f$ itself.
\end{proof}

\begin{corollary} The categories $\Ban$, $\Born_\C$, $\Born_\C^s$ and $\CBorn_\C$ 
are not abelian.
\end{corollary}
\begin{proof} By Proposition \ref{prop:ab-epimono=iso}, it suffices 
to exhibit an arrow in each of the categories in question which is both
a monomorphism and an epimorphism but not an isomorphism.

In $\Born_\C$, $\Born_\C^s$ and $\CBorn_\C$, 
we may take the identity function of an infinite dimensional
Banach space $X$ viewed as an arrow $\Cpt(X) \to \vN(X)$.
To see that this is not an isomorphism note
for example that the open unit ball of $X$ is von Neumann
bounded but not precompact.

For $\Ban$ we may consider the inclusion $\ell^1 \to c_0$. This map
is injective with dense range, and so it is both a monomorphism
and an epimorphism. However, 
it is not bijective and thus it cannot be an isomorphism.
\end{proof}

As the result above shows, if we wish
to do homological algebra in our setting 
we will need to consider a less restrictive
notion than that of an abelian category. 
We say that $\mathsf C$ is \emph{quasi-abelian}
if strict monomorphisms are stable under pushouts
and struct epimorphisms are stable under pullbacks.
Explictly, this means that for every square
\begin{equation}\label{diag:quasiab}
\begin{tikzcd}
X \arrow{r}{f} \arrow{d} & Y \arrow{d} \\
Y' \arrow{r}{g} & Z
\end{tikzcd}
\end{equation}
we have that:
\begin{itemize}
    \item[(i)] if \eqref{diag:quasiab} is a pullback square and $g$ is a strict epimorphism, then so is $f$.
    \item[(ii)] if \eqref{diag:quasiab} is a pushout square and $f$ is a strict monomorphism, then so is $g$.
\end{itemize}


\bigskip
\exs % This command prints "Excersices for Section ##"
\bigskip
\begin{exercise} \label{ex:disk-sum-iso}
Complete the proof that 
the linear map $V_{D_1} \oplus V_{D_2} / \ker(q)
\to V_{D_1+D_2}$ of Lemma \ref{lem:bornologies-directed} is an isometry.
\end{exercise}

\begin{exercise} \label{ex:disk-int}
Prove that the intersection of two (complete) disks 
is again a (complete) disk.
\end{exercise}

\begin{exercise}\label{ex:int-bor}
Let $V$ be a vector space and $(\mathfrak{B}_\alpha)_{\alpha \in \Lambda}$ a
family of bornologies on $V$. 
Prove that the intersection $\bigcap_{\alpha \in \Lambda} \mathfrak B_\alpha$
is a bornology on $V$.
\end{exercise}

\begin{exercise} \label{ex:circ-conv=disk}
Prove the chain of equalities \eqref{eq:Sd=Scc}.
\end{exercise}

\begin{exercise} \label{ex:fine}
Prove that if $V$ and $W$ are 
two vector spaces equipped with
the fine bornology, then 
any linear map $T \colon V \to W$
is bounded. Moreover, show that this defines a fully faithful functor $\mathsf{Fine} \colon \mathsf{Vect}_{\C} \to \mathsf{Born}_\C$.
\end{exercise}

\begin{exercise} \label{ex:equib-op}
Let $V$ and $W$ be two Banach spaces.
Prove that the equibounded bornology
on $\Hom(V,W)$ coincides with 
the von Neumann bornology induced
by the operator norm.
\end{exercise}

\begin{exercise} \label{ex:conv-Cpt=vN}
Prove that in a Frechét space $V$
a sequence $(x_n)_{n \in \N}$
converges to a point $x \in V$
for the von Neumann bornology 
if and only if it converges for
the precompact bornology.
\end{exercise}

\begin{exercise} \label{ex:tensor-hom}
Prove Theorem \ref{thm:tensor-hom}.
\end{exercise}
\bigskip

\section{Bornological functional analysis in the nonarchimedean case}

\chapter{Abstract models for bornologies}



\section{placeholder}

At the level of Banach spaces, continuity is the same as boundedness. We enlarged this category by taking its pro-completion, but problems persist if we care about say the topology on the mapping space. How about taking its ind-completion and working instead with bounded maps? This leads to the theory of bornological vector spaces. 

\subsection{Bornological vector spaces over \(\R\) and \(\C\)}



\begin{example}[Fine bornology]
Finite dimensional subspaces of a vector space. It is the smallest bornology on a vector space.
\end{example}

\begin{example}[Von Neumann bornology]
Let \(V\) be a locally convex topological vector space. A subset \(B\subseteq V\) is \emph{von-Neumann bounded} if \(\nu(B) \subseteq \R_{>0}\) for each semi-norm \(\nu\). Recovers `usual' boundedness.  
\end{example}

\begin{example}[Pre-compact bornology]
Let \(V\) be a locally convex topological vector space. A subset \(B \subseteq V\) is \emph{pre-compact} if for every neighbourhood \(U\) of \(0\), there is a finite subset \(F \subseteq V\) such that \(B \subseteq U + F\).  
\end{example}



\begin{itemize}
\item We like to work with \emph{complete} topological or bornological vector spaces (eg: Banach spaces). Define completion for bornological vector spaces.
\item If the space is metrisable, then topological convergence and Cauchyness are equivalent to bornological convergence and Cauchyness (in the von Neumann and precompact bornologies);
\item For a metrisable space, topological completeness is equivalent to completeness in the von Neumann and precompact bornology.
\end{itemize}


\subsection{Bornological vector spaces in the nonarchimedean setting}


Copy-paste \cite{Cortinas-Cuntz-Meyer-Tamme:Nonarchimedean}*{Section 2}


\subsection{Abstract models for bornologies}


\begin{theorem}[Meyer, Meyer-Mukherjee]
The category of complete bornological vector spaces over \(\R\), \(\C\) or \(\Q_p\) embeds into the category of inductive system of Banach spaces. The essential image of this functor is the category of inductive systems of Banach spaces with injective structure maps.
\end{theorem}

Viewing complete bornological vector spaces as certain inductive systems of Banach spaces, we can therefore reinterpret the bornological theory using categorical operations on the category of Banach spaces. 

\begin{definition}
Let \(R\) be a Banach ring. The category of complete bornological \(R\)-modules is defined as the category of inductive systems of Banach \(R\)-modules with injective structure maps.
\end{definition}


\begin{theorem}
The above category has great properties.
\end{theorem}


\textbf{Mention application in analytic geometry.}


\subsection{Internal closure}


We start with a remark. If $(V,p)$ is a 
semi-normed space with open and closed unit balls $B_1$ and $B_2$ respectively, 
then any disk $B_1 \subset D \subset B_2$ gives rise to 
the same gauge semi-norm in $V$. Thus, 
the bijectivity of correspondence between
disks and semi-norms breaks down for disks that are not internally closed.

\begin{definition}
Let $V$ be a vector space and $S \subset V$ a subset of $V$. 
Its \emph{internal closure} $S^\intc$ is defined as the minimal
internally closed set containing $S$.
\end{definition}

\begin{lemma} Let $V$ be a vector space. If $D \subset V$ is a disk, then: 
\begin{itemize}
\item[(i)] $D^\intc = \{x \in V : rx \in D  \ (\forall r \in [0,1))\}$;
\item[(ii)] $D^\intc$ is a disk and $V_D = V_{D^\intc}$; 
\item[(iii)] $D^\intc$ is the closed unit ball of $p_D$;
\item[(iv)] $p_{D^\intc} = p_D$;
\item[(v)] if $D$ is complete, then so is $D^\intc$.
\end{itemize}
\end{lemma}
\begin{proof} Put $X = \{x \in V : rx \in D  \ (\forall r \in [0,1))\}$.
Given $x \in D$ and $r \in [0,1)$, by hypothesis the convex combination $rx = rx+(1-r)0$ lies in $D$, so $D \subset X$. 
Suppose now that $x \in V$ is such that $sx \in X$
for any $s \in [0,1)$. Then, for any $r \in [0,1)$, we have $rsx \in D$.
and in particular $rx = (\sqrt{r}\sqrt{r}) x$ lies in $D$ for any $r \in [0,1)$. 
This shows that $x \in D^\intc$ and thus that $X$ is internally closed. To conclude 
(i), note that if $Y \supset D$ is internally closed and $x \in X$, then $rx \in D \subset Y$ for all $r \in [0,1)$; hence $x \in Y$.

From (i) it follows that $D^\intc$ is circled if $D$ is so.
The rest of the assertions follow from the fact that for a given disk $D$, the closed unit ball of $V_D$ is an internally closed subset of $V$ containing $D$ and moreover it is minimal in this regard.
\end{proof}


\subsection{Disks again}

Let \(V\) be a bornological vector space. Consider a bounded subset \(B \in \mathcal{B}(V)\). By the axioms of a bornology, \(B\) is contained in an \textbf{internally closed} disk \(D \in \mathcal{B}(V)\), that is, \(D\) is a bounded internally closed disk. 

\begin{definition}
    The disked hull \(B^\diamond\) of \(B\) is the intersection of all bounded, internally closed disks containing \(B\).
\end{definition}

Note that the above intersection is nonempty axiomatically. To compute it explicitly, we have the following:

\begin{lemma}
    \(B^\diamond \) is the internal closure of the set \(\setgiven{\sum_{i=1}^n \lambda_i x_i}{x_i \in B, \sum_{i=1}^n \abs{\lambda_i}\leq 1}\).
\end{lemma}

\begin{proof}
    First of all, \(B^\diamond\) is a disk. It is also internally closed. \textbf{The minimality follows by construction, I suppose.}
\end{proof}

Now consider a complete bornological vector space \(V\). Then by axiom, any bounded subset \(B\) is contained in a complete, internally closed, bounded disk. 

\begin{definition}
The completed disked hull \(B^{\heartsuit}\) is the internal closure of the intersection of all complete, internally closed bounded disks containing \(B\).
\end{definition}

\begin{lemma}
    \(B^\heartsuit = \setgiven{\sum_{n=0}^\infty \lambda_n x_n}{x_n \in B, \sum_{n \in \N} \abs{\lambda_n} \leq 1}^\intc\).
\end{lemma}

\begin{proof}
    \textbf{Check}.
\end{proof}


\begin{theorem}
    The categories \(\mathsf{D}(\mathsf{CBorn})\) and \(\mathsf{D}(\mathsf{Ind}(\mathsf{Ban}))\) are equivalent.
\end{theorem}

\begin{proof}
    Let \(X \in \mathsf{Ind}(\mathsf{Ban})\) be an inductive system of Banach spaces. Its separated inductive limit is a complete bornological vector space, so we get a functor \[\mathsf{Ind}(\mathsf{Ban}_\C) \to \mathsf{CBorn}_\C.\] This functor is additive, so it extends to a functor on the homotopy categeory of chain complexes \[\mathsf{Kom}(\mathsf{Ind}(\mathsf{Ban})) \to \mathsf{Kom}(\mathsf{CBorn}_\C).\] \textbf{It can be shown} that this functor also Kan extends to the derived categories, so we get a functor in one direction. In the other direction, take a chain complex \(X\) in \(\mathsf{D}(\mathsf{CBorn})\). For a complete bornological vector space \(X\), define \(Q(X)\) as the inductive system \((\comb{X_i})_{i \in I}\), where \(\mathsf{diss}(X) = (X_i)\). In other words, this is the completion taken in the category of inductive systems of Banach spaces. This is an exact functor, so it extends to derived categories. The functors \(Q\) and \(R\varinjlim\) are inverse to each other.  
\end{proof}