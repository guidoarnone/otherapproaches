
\subsection{The projective bornological tensor product}

In this section, we introduce the \textit{bornological tensor product}. Let \(V\) and \(W\) be bornological vector spaces. We define the \textit{(projective) tensor product bornology} as the bornology generated by subsets \(B_1 \otimes B_2\), where \(B_1\) and \(B_2\) are bounded disks in \(V\) and \(W\). In other words, a subset is bounded if and only if it is contained in the disked hull of \(B_1 \otimes B_2\) for \(B_1 \subseteq V\) and \(B_2 \subseteq W\) bounded disks. The bornological tensor product satisfies the universal property that for any bounded bilinear map \(f \colon V \times W \to Z\) into a bornological vector space, there is a unique bounded linear map \(\tilde{f} \colon V \otimes W \to Z\) such that \(\tilde{f} \circ b = f\), where \(b \colon V \times W \to V \otimes W\) is the canonical map.

Now let \(V\) and \(W\) be complete, bornological vector spaces. Their \textit{completed bornological tensor product} \(V \haotimes W\) is defined as the (yet to be defined) bornological completion \[V \haotimes W = \comb{(V \otimes W)}\] of the tensor product \(V \otimes W\) with respect to the tensor product bornology. 

We now compare the bornological tensor product with the topological tensor product, or more precisely, the projective topological tensor product. Recall that the completed projective topological tensor product \(V \haotimes_{\pi} W\) of two  complete, locally convex topological vector spaces is defined as complete locally convex topological vector space representing jointly continuous bilinear maps \(V \times W \to X\) into a complete, locally convex topological vector space. In this direction, we have the following:


\begin{theorem}\label{thm:Grothendieck-tensor}
Let \(V\) and \(W\) be Frech\'et spaces. The canonical continuous bilinear map \(b \colon V \times W \to V \haotimes_{\pi} W\) induces a bornological isomorphism \(\mathsf{Cpt}(V) \haotimes \mathsf{Cpt}(W) \cong \mathsf{Cpt}(V \haotimes_{\pi} W)\). 
\end{theorem}

To prove this, we will use the following theorem (Grothendieck):
\begin{theorem}\label{thm:Grothendieck-x-as-a-serie}
Let \(V\) and \(W\) be Frechét spaces. Any \(x \in V \haotimes_{\pi} W\) is of the form \[x = \sum_{n \in \N} \lambda(n) y_n \otimes  z_n \] with null sequences \((y_n) \in V\), \((z_n) \in W\) and \(\lambda \in l^1(\N)\). 

Furthermore, if \((x_k)_{k\in \N}\) is a null sequence in \(V \haotimes_{\pi} W\), then there are null sequences  \((y_n)\) in \(V\) , \((z_n) \) in \(W\) ; and a null sequence \(\lambda_k \in l^1(\N)\) with \[x_k = \sum_{n \in \N} \lambda_k (n) y_n \otimes  z_n \]
\end{theorem}

A proof could be found in \cite{grothendieck}*{pages 51 and 57}.

\begin{proof}
Let \(X\) be a complete bornological vector space. We want to show that a bounded bilinear map \(\mathsf{Cpt}(V) \times \mathsf{Cpt}(W) \to X\) extends uniquely to a bounded linear map \(\mathsf{Cpt}(V \haotimes_{\pi} W)\). Since \(\mathsf{Cpt}(V) \haotimes \mathsf{Cpt}(W)\) is universal for this property, the result will then follow from the Yoneda Lemma. 

Let \(f\colon \mathsf{Cpt}(V) \times \mathsf{Cpt}(W) \to X\) be a bounded bilinear map. We want to show that there is a unique bounded linear map \(\tilde{f}\colon \mathsf{Cpt}(V \haotimes_{\pi} W) \to X\) such that \(\tilde{f} \circ b = f\). Suppose such a map exists, then 
\[\tilde{f}(x) = \sum_{n \in \N} \lambda_n \tilde{f}(y_n \otimes  z_n) = \sum_{n \in \N} \lambda_n f(y_n, z_n),\] where we have used that any element \(x \in V \haotimes_{\dvr} W\) can be written as a series \(x = \sum_{n \in \N} \lambda_n y_n \otimes z_n\), where \(\lambda \in l^1(\N)\), \((y_n) \in V\) and \((z_n) \in W\) are null sequences as stated by the theorem \ref{thm:Grothendieck-x-as-a-serie}. As a consequence, if such an \(\tilde{f}\) exists, it is completely determined by \(f\) and hence unique. 

Now define \(\tilde{f} \colon V \haotimes_{\pi} W \to X\) as \[\tilde{f}(x) = \sum_{n \in \N} \lambda_n f(y_n,z_n),\] where \(x = \sum_{n \in \N} \lambda_n y_n \otimes z_n,\) for null sequences \((y_n) \in V\), \((z_n) \in W\) and \(\lambda \in l^1(\N)\). We need to show that this formula is independent of the choices of the representing sequences \((y_n)\) and \((z_n)\). It suffices to show that for \(0 = \sum_{n \in \N} \lambda_n y_n \otimes z_n\), we have \(\tilde{f}(0) = 0\). Let \(x_k \defeq \sum_{n \leq k} \lambda_n y_n \otimes z_n\).  Then by hypothesis, \(\lim_{k \to \infty} x_k = 0\), so that \(x_k\) is a null-sequence in \(V \haotimes_{\pi} W\). By theorem  \ref{thm:Grothendieck-x-as-a-serie}, there are null sequences \(\lambda_{k}' \in l^1(\N)\) and \((y_n')\) and \((z_n')\), such that \(\sum_{n \leq k} \lambda_n y_n \otimes z_n = x_k = \sum_{n \in \N}\lambda_{k,n}' y_n' \otimes z_n' \in V \haotimes_{\pi} W\). Since \(f\) is a bounded bilinear map, the subset \(\setgiven{f(y_n',z_n')}{n \in \N}\) is bounded in \(X\). Consequently, \[\lim_{k \to \infty} \sum_{n \in \N} \lambda_{k,n}' f(y_n',z_n') = 0.\] The term inside the limit above is \(\tilde{f}\) applied to \(x_k\). But applying \(\tilde{f}\) to \(x_k\) is also \(\sum_{n \leq k} \lambda_n f(y_n, z_n)\), whose limit as \(k \to \infty\) is \(\sum_{n \in \N} \lambda_n f(y_n, z_n)\). Summarily, \[\tilde{f}(0) = \sum_{n \in \N} \lambda_n f(y_n,z_n) = \lim_{k \to \infty} \sum_{n \in \N} \lambda_{k,n}'f(y_n', z_n') = 0,\] as required. 

Finally, it remains to show that \(f \colon \mathsf{Cpt}(V \haotimes_\pi W) \to X\) is bounded. Suppose \(S \subseteq V \haotimes_{\pi} W\) is precompact. Then by \textbf{cite Grothendieck}, \(S\) is contained in the complete disked hull of a null sequence \((x_n)\). \textbf{Note that we don't really need the explicit description of the complete disked hull here.} Consequently, the complete disked hull of \(\setgiven{\tilde{f}(x_n)}{n \in \N}\) is bounded in \(X\), as required. 
\end{proof}


What we have shown therefore is that there is a well-behaved notion of a tensor product in the bornological framework, which generalises the completed projective topological tensor product for Frech\'et spaces. But we get more in this framework, namely, 

\begin{theorem} \label{thm:tensor-hom}
    For (separated) bornological vector spaces \(V\) and \(W\), we have \(\Hom(V, \underline{\mathsf{Hom}}(W,Z)) \cong \Hom(V \otimes W, Z)\), where \(\otimes\) denotes the bornological tensor product. For complete bornological vector spaces \(V\) and \(W\), we have \(\Hom(V, \underline{\mathsf{Hom}}(W,Z)) \cong \Hom(V \haotimes W, Z),\), where \(\haotimes\) is the completed projective tensor product. In other words, we have a tensor-Hom adjunction. 
\end{theorem}
\begin{proof} Left as Exercise \ref{ex:tensor-hom}. 
\end{proof}