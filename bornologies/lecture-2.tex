
In the proof of Lemma \ref{lem:bornologies-directed}, we used that bornologies are closed under convex hulls. We also desire the same for bounded disks and complete, bounded disks. This is clear as \(\mathcal{B}_d(V)\) and \(\mathcal{B}_c(V)\) are both closed under arbitrary intersections (Exercise \ref{ex:disk-int}). We call the minimal disk (respectively, complete disk) containing a bounded subset (respectively, complete disk) the \emph{disked hull}, (respectively, \emph{complete disked hull}) 
of the given bounded subset.
Given any subset
$S \subset V$, its \emph{circled hull} is $S^{\circ} := \bigcup_{|\lambda| \le 1} \lambda S$. This is the minimal circled subset containing $S$. Likewise, 
recall that the convex hull of $S$ is 
the minimal convex set that contains $S$ and can be defined as $\conv(S) = \{\sum_{i=1}^n \lambda_i s_i : s_i \in S, \lambda_i \in \R_{>0}, \sum_{i=1}^n \lambda_i = 1\}$.
We can characterize 
the disked hull $S^\dhull$ of $S$ as
\begin{equation}\label{eq:Sd=Scc}
    S^\dhull = \conv(S^\circhull) = \left\{\sum_{i=1}^n \lambda_i s_i : s_i \in S, \lambda_i \in \C, \sum_{i=1}^n |\lambda_i| \le 1\right\}.
\end{equation}
The verification of the equality above
is the content of Exercise \ref{ex:circ-conv=disk}.

We can now finally look at some examples of bornologies.

\begin{example}
    The most basic example of a bornology is the \emph{fine bornology}. Its elements are subsets \(S \subseteq V\) that are contained in the disked hull of finite sets.  
\end{example}

It is sometimes convenient and economical to describe bornologies using a generating set of bounded subsets. To turn such a generating set into a bornology, we first notice that given a vector space, the power set \(\mathcal{P}(V)\) of \(V\) is always a bornology on \(V\). Consequently, if \(\mathcal{B}\) is a set of subsets of \(V\), we can always define the nonempty intersection 
\[ \gen{\mathcal{B}} \defeq \bigcap_{\mathcal{B}' \supseteq \mathcal{B}} \mathcal{B}'\] of 
bornologies containing \(\mathcal{B}\). This is itself a bornology, 
called the \emph{bornology generated by \(\mathcal{B}\)}; we leave the verification of this fact
as Exercise \ref{ex:int-bor}.

\begin{definition}\label{def:fine-coarse}
Given two bornologies $\mathcal B \subset \mathcal B'$, we say that $\mathcal B'$ is \emph{finer} than $\mathcal B$
or that  $\mathcal B$ is \emph{coarser} than $\mathcal B'$.
\end{definition}

Using the terminology defined above, the bornology generated by a set can be characterized as the coarsest 
bornology that contains it.


\begin{example}
    We now describe our first honest example of a bornology. Given a locally convex topological vector space \(V\) with a family of semi-norms \(p_i \colon V \to \R_{\geq 0}\), we define the \emph{von Neumann bornology} as the bornology where a subset \(B \subseteq V\) is bounded if \(p_i(B)\) is a bounded subset of \(\R\) for each \(i\). Equivalently, \(B \subseteq_a B_{p_i}\) for each \(i\). We shall denote this bornology by \(\vN(V)\).
\end{example}

\begin{example}
    Let \(V\) be a locally convex topological vector space. A subset \(S \subseteq V\) is called \emph{precompact} if for every neighbourhood \(U\) of the origin, there is a finite set \(F\) such that \(S \subseteq F + U\). These precompact sets form a bornology on \(V\), which we denote by \(\Cpt(V)\). To see why any such precompact set is contained in a precompact disk, it suffices to show that the disked hull of a precompact set is precompact. Clearly \(\Cpt(V)\) is closed under scalar multiplication and addition. Consequently, \(t S_1 + (1-t)S_1\) is precompact whenever \(S_1\) is, for every \(t\in [0,1]\). So it suffices to show that the circled hull of a precompact set is precompact. Let \(S \subseteq F + U\), where \(U\) is a circled neighbourhood of the origin. Then the circled hull of \(S\) is contained in \(F^\circhull + U\), where \(F^\circhull\) is the circled hull of \(F\). Consequently, it suffices to show that \(F^\circhull\) is precompact. Since precompact subsets are closed under finite unions, we are reduced to the case where \(F\) is a singleton \(\{x\}\). But then \(F^\circhull\) is the image of the unit disk in \(\C\) under the (continuous) linear map \(\C \to V\), \(\lambda \mapsto \lambda x\). The conclusion now follows from the fact that the continuous image of a precompact set under a linear map is precompact. 
\end{example}

\begin{proposition} \label{prop:Cpt-coarser-vN} Let $V$ be a locally convex topological
vector space. The proecompact bornology is coarser than the von Neumann bornology. 
\end{proposition}
\begin{proof} Let $S \subset V$ be a precompact subset and let us see that it is von Neumann bounded.
This amounts to proving that, given a seminorm $p_i \colon V \to \R_{\ge 0}$, 
the subset $S$ is absorbed by the unit ball $B := B_{p_i}$ of $p_i$. Given that $S$ is precompact
and $B$ is a neighbourhood of the origin, we know that there must exist a finite set 
$F = \{x_1, \ldots, x_k\}$ such that 
\[
S \subset B + F = B_{p_1}(x_1,1) \cup \cdots \cup B_{p_1}(x_n, 1).    
\]
It remains to note that, setting $D = \max_{1 \leq k \leq n} p_i(x_k)$, 
\[
    B_{p_1}(x_1,1) \cup \cdots \cup B_{p_1}(x_n, 1)
    \subset (D+1)B. 
\]
\end{proof}

\subsection{Bounded maps}

We now move to the morphism space in the world of bornologies. Unsurprisingly, we call a linear map \(T \colon V \to W\) \textit{bounded} if for every bounded subset \(S \subseteq V\), the image \(T(S)\) is bounded in \(W\). 
We denote the set of bounded linear maps from \(V\) to \(W\) by \(\Hom(V,W)\). A subset $L \subset \Hom(V,W)$
is \emph{equibounded} if for every 
bounded subset \(B \subseteq V\), the set \(L(B) \defeq \setgiven{T(x)}{T \in L, x \in B}\) 
is bounded in \(W\).

\begin{lemma} \label{lem:equi-born}
    Equibounded subsets form a bornology on \(\Hom(V,W)\).
\end{lemma}
\begin{proof}
    Let \(L\) be an equibounded family of maps, 
    and let \(T \in L^\circhull\) be an element of the circled hull of \(L\). 
    Then there is an \(S \in L\) and a \(\lambda \in \C\) with 
    \(\abs{\lambda} \leq 1\) such that \(T = \lambda S\). Consequently, 
    \(\conv(L)(B) \subseteq L(B)\) for all bounded subsets 
    \(B\) in \(V\), so that \(\conv(L)\) is equibounded as well. 
    The same line of reasoning shows that the given collection of 
    equibounded maps is closed under inclusions. Finite unions are trivial to check. 
    It remains to see that the collection of equibounded linear maps is closed under 
    taking disked hulls. Again, let \(L\) be an equibounded set of linear maps. Then its
    disked hull satisfies \(L^\dhull(B) \subseteq L^\dhull(D) \subseteq L(D)\) for any 
    bounded subset \(B\) contained in a bounded disk \(D\).  
\end{proof}

We denote by \(\mathsf{Born}_\C\), \(\mathsf{Born}_\C^s\) and \(\mathsf{CBorn}_\C\) the categories 
of bornological vector spaces, separated bornological vector spaces and complete bornological vector 
spaces with bounded linear maps (endowed with the equibounded bornology), respectively. After we have 
developed some analysis, we will prove the following:

\begin{lemma}
    If \(W\) is separated (respectively, complete), then \(\Hom(V,W)\) with the equibounded bornology is separated (respectively, complete). 
\end{lemma}