\chapter{Bornologies}

As mentioned in the introduction, the category of Banach spaces \(\mathsf{Ban}\) is the smallest meaningful category in which functional analysis combines well with homological algebra. Specifically, it is a closed, symmetric monoidal category with respect to the completed projective tensor product and the Hom-space equipped with the operator norm.  However, the category of Banach spaces is not closed under, say, projective limits, so that important algebras such as \(\Cont^\infty(M)\) are not included in it. There are two ways of remedying this -- working with the \emph{pro-completion} and the \emph{ind-completion}. The former category includes complete, locally convex topological vector spaces, but still lacks an internal Hom functor that is right adjoint to the completed projective tensor product. What we therefore instead look at is the category \(\mathsf{Ind}(\mathsf{Ban})\) of inductive systems of Banach spaces. It is also worth noting that so far we have not fixed a base field over which we propose to do functional analysis -- this is manifestly deliberate. We wish to develop a framework that yields desirable categorical outcomes no matter what the base field or ring may be.  \textbf{Write Leitfaden for chapter.}


\section{Concrete models for bornologies}

The category of inductive systems \(\mathsf{Ind}(\mathsf{Ban})\), while having useful categorical properties that we will eventually see, has the drawback that it is not a concrete category. It turns out, however, that it is equivalent (in a sense that will be made precise) to a concrete category. This category can alternatively be described by objects that are vector spaces with additional structure, axiomatically referred to as ``bounded sets''. So far, this alternative characterisation has been explicitly worked out in the case where the base field is \(\R\), \(\C\) with their Euclidean topologies (colloquially referred to as ``Archimedean'' fields) and a complete, discrete valuation field \(\dvf\) (such as \(\Q_p\)) with its usual ``nonarchimedean'' valuation.  

\subsection{Bornological functional analysis in the archimedean case}

Throughout this section, we work with vector spaces over \(\R\) and \(\C\). We first start with an investigation of the internal structure of topological vector spaces. More concretely, if we start with a vector space \(V\) with a semi-norm \(\rho \colon V \to \R_{\geq 0}\), we can define a subset \[B \defeq \setgiven{x \in V}{\rho(x) \leq 1},\] called the \emph{unit ball} of \(\rho\). This subset satisfies the following properties:

\begin{enumerate}
    \item For all scalars \(\lambda \in \C\) with \(\abs{\lambda} \leq 1\), we have \(\lambda \cdot B \subseteq B\);
    \item For all \(x\), \(y \in B\), \(tx + (1-t)y \in B\) for all \(t \in [0,1]\);
    \item If for all \(r \in [0,1)\) we have \(rx \in B\), then \(x \in B\);
    \item \(\R_{\geq 0} \cdot B = V\).
\end{enumerate}

\begin{definition}
    A subset \(D \subseteq V\) of a vector space is called a \emph{disk} if it satisfies the first two properties above. We call a disk \textit{internally closed} if it satisfies the third property. A disk is said to be \emph{absorbent} if it satisfies the fourth property. 
\end{definition}


What is insightful is that every semi-norm can be recovered from an internally closed, absorbent disk, as the following theorem reveals:

\begin{theorem}
   Let \(D \subseteq V\) be an internally closed, absorbent disk. There exists a semi-norm \(\rho \colon V \to \R_{\geq 0}\) whose closed unit ball is \(D\). 
\end{theorem}

\begin{proof}
    The required semi-norm is given by 
    \[\rho \colon V \to \R_{\geq 0}, \qquad \rho(x) \defeq \inf \setgiven{\lambda \geq 0}{x \in \lambda D}.\]
    To begin with, this assignment is well-defined because if \(x \in V\), then since \(D\) is absorbent, there really is a \(\lambda \in \R_{\geq 0}\) and a \(v \in V\) such that \(x = \lambda v\). To see that it is scale-invariant, let \(\lambda \in \C\) and \(x \in V\). If \(\lambda = 0\), then this is clear, so let us assume otherwise. 
    Moreover, since $S^1 \cdot D = D$, it follows that $\rho(\omega x) = \rho(x)$ for all $\omega \in S^1$, $x \in V$
    and thus we may assume $\lambda \in \R_{> 0}$.
    We then have \(\rho(\lambda x) = \inf \setgiven{\alpha \geq  0}{\lambda x \in \alpha D} = \inf \setgiven{\alpha \geq 0}{x \in \frac{\alpha}{\lambda} D }\) and
    \begin{align*}
        \lambda \rho(x) &= \lambda\inf \setgiven{\alpha \geq 0}{x \in \alpha D} \\
        & =\inf \setgiven{\beta = \lambda\alpha }{x \in \alpha D} = \inf\left\{\beta \ge 0 :  x \in \frac{\beta}{\lambda} D \right\},
    \end{align*} as required. Finally, to show the triangle inequality, we use that \(D\) is absorbent to find \(\alpha\) and \(\beta\) such that \(x \in \alpha D\) and \( y \in \beta D\). Therefore, \(x + y \in \alpha D + \beta D = (\alpha + \beta) D,\) where the second equality follows from convexity. Consequently, \(\rho(x + y) \leq \alpha + \beta\). Since \(\alpha\) and \(\beta\) are arbitrary, we have \(\rho(x+ y) \leq \rho(x) + \rho(y)\). That the unit ball of this semi-norm is \(B\) is clear.
\end{proof}

The semi-norm in the proof of theorem above is called the \emph{gauge semi-norm} on a disk. Note that if \(D\) is a not necessarily absorbent disk in \(V\), then we can make it absorbent in the subspace \(V_D \defeq \R_{\geq 0} \cdot D\). The latter when equipped with the gauge semi-norm on \(D\) becomes a semi-normed space inside \(V\), which in itself is just a vector space. We shall call a disk \(D\) \emph{norming} if \(V_D\) with the gauge semi-norm is actually a normed space; we call it \emph{complete} if \(V_D\) is completed and normed, that is, a Banach space. In conclusion, we have in some sense created a topological vector space from a disk, which is just a subset of a vector space with some axioms. This is exactly what we set out to do.


\begin{definition}[Bornology]
A \emph{bornology} on a vector space \(V\) is a collection \(\mathfrak{B}\) of its subsets (called \emph{bounded subsets}) satisfying the following natural properties:

\begin{itemize}
\item \(\{x\} \in \mathfrak{B}\) for all \(x \in V\);
\item if \(S \subseteq T\) and \(T \in \mathfrak{B}\), then \(S \in \mathfrak{B}\);
\item \(S \cup T \in \mathfrak{P}\) whenever \(S\), \(T \in \mathfrak{B}\);
\item if \(S \in \mathfrak{B}\), then \(r S \in \mathfrak{B}\) for \(r>0\);
\item any subset \(S \in \mathfrak{B}\) is contained in a disk \(T \in \mathfrak{B}\). 
\end{itemize}


\end{definition}

We call a bornological vector space \emph{separated} if every bounded subset is contained in a norming disk; we call it \emph{complete} if every bounded subset is contained in a complete, bounded disk. Denote by \(\mathcal{B}(V)\) the collection of bounded subsets of a bornological vector space \(V\). We also have two other related collections, namely, the collections \(\mathcal{B}_d(V)\) and \(\mathcal{B}_c(V)\) of bounded disks, and complete bounded disks. These collections carry two canonical preorders, namely, inclusion and \emph{absorption}, where we say \(S \subseteq_a T\) if there exists a \(c>0\) such that \( S \subseteq c T\). Recall that sets with preorders \((S, \leq)\) form a category, where objects are given by elements of the set, and there is a unique morphism \(x \to y\) if and only if \(x \leq y\). We want the preordered sets \(\mathcal{B}(V) \supseteq \mathcal{B}_d(V) \supseteq \mathcal{B}_c(V)\) to carry all the topological information in a topological vector space, in the same sense that a sequence of seminorms carries all the information about, say, a Frechet space. To make this more precise, we will need that these preordered sets viewed as categories are directed sets:

\begin{lemma}\label{lem:bornologies-directed}
The categories \(\mathcal{B}(V)\), \(\mathcal{B}_d(V)\) and \(\mathcal{B}_c(V)\) are directed. Furthermore, the subset \(\mathcal{B}_d(V)\) is cofinal in \(\mathcal{B}(V)\). The subset \(\mathcal{B}_c(V)\) is cofinal in \(\mathcal{B}(V)\) if and only if \(V\) is complete.  
\end{lemma}

\begin{proof}
    Let \(D_1\) and \(D_2\) be bounded subsets of \(V\). Then \(D_1 + D_2\) is contained in the convex hull of \(2D_1 \cup 2D_2\), which is bounded as bornologies are closed under multiplication by positive real numbers, taking finite unions and convex hulls. Consequently, \(D_1 + D_2\) is bounded, as bornologies are hereditary for inclusions. This shows that \(\mathcal{B}(V)\) is directed. To see that \(\mathcal{B}_d(V)\) is directed, we need that \(D_1 + D_2\) is a disk, whenever \(D_1\) and \(D_2\) are bounded disks, and this is easy to see. Finally, if \(D_1\) and \(D_2\) are complete, bounded disks, then we need to show that \(D_1 + D_2\) is complete. It suffices to show that \(V_{D_1 + D_2}\) is a Banach space. To see this, consider the canonical map \(q \colon V_{D_1} \oplus V_{D_2} \to V_{D_1 + D_2}\) taking \((x,y) \mapsto x + y\). Then there is an isometric isomorphism \(V_{D_1} \oplus V_{D_2}/\ker(q) \cong V_{D_1 + D_2}\), where the domain has the quotient norm. We leave the 
    details as Exercise \ref{ex:disk-sum-iso}.
\end{proof}

In the proof of Lemma \ref{lem:bornologies-directed}, we used that bornologies are closed under convex hulls. We also desire the same for bounded disks and complete, bounded disks. This is clear as \(\mathcal{B}_d(V)\) and \(\mathcal{B}_c(V)\) are both closed under arbitrary intersections (Exercise \ref{ex:disk-int}). We call the minimal disk (respectively, complete disk) containing a bounded subset (respectively, complete disk) the \emph{disked hull}, \emph{complete, disked hull} of the given bounded subset.
Given any subset
$S \subset V$, its \emph{circled hull} is $S^{\circ} := \bigcup_{|\lambda| \le 1} \lambda S$. This is the minimal circled subset containing $S$. Likewise, 
recall that the convex hull of $S$ is 
the minimal convex set that contains $S$ and can be defined as $S^c = \{\sum_{i=1}^n \lambda_i s_i : s_i \in S, \lambda_i \in \R_{>0}, \sum_{i=1}^n \lambda_i = 1\}$.
We can characterize 
the disked hull $S^d$ of $S$ as
\begin{equation}\label{eq:Sd=Scc}
    S^d = (S^\circ)^c = \left\{\sum_{i=1}^n \lambda_i s_i : s_i \in S, \lambda_i \in \C, \sum_{i=1}^n |\lambda_i| \le 1\right\}.
\end{equation}
The verification of the equality above
is the content of Exercise \ref{ex:circ-conv=disk}.

We can now finally look at some examples of bornologies.

\begin{example}
    The most basic example of a bornology is the \emph{fine bornology}. Its elements are subsets \(S \subseteq V\) that are contained in the disked hull of finite sets.  
\end{example}

It is sometimes convenient and economical to describe bornologies using a generating set of bounded subsets. To turn such a generating set into a bornology, we first notice that given a vector space, the power set \(\mathcal{P}(V)\) of \(V\) is always a bornology on \(V\). Consequently, if \(\mathcal{B}\) is a set of subsets of \(V\), we can always define the nonempty intersection 
\[ \gen{\mathcal{B}} \defeq \bigcap_{\mathcal{B}' \supseteq \mathcal{B}} \mathcal{B}'\] of bornologies containing \(\mathcal{B}\). This is itself a bornology, called the \emph{bornology generated by \(\mathcal{B}\)}; we leave the verification as Exercise \ref{ex:int-bor}.


\begin{example}
    We now describe our first honest example of a bornology. Given a locally compact topological vector space \(V\) with a family of semi-norms \(p_i \colon V \to \R_{\geq 0}\), we define the \emph{von Neumann bornology} as the bornology where a subset \(B \subseteq V\) is bounded if \(p_i(B)\) is a bounded subset of \(\R\) for each \(i\). Equivalently, \(B \subseteq_a B_{p_i}\) for each \(i\). We shall denote this bornology by \(\vN(V)\).
\end{example}

\begin{example}
    Let \(V\) be a locally convex topological vector space. A subset \(S \subseteq V\) is called \emph{precompact} if for every neighbourhood \(U\) of the origin, there is a finite set \(F\) such that \(S \subseteq F + U\). These precompact sets form a bornology on \(V\), which we denote by \(\Cpt(V)\). To see why any such precompact set is contained in a precompact disk, it suffices to show that the disked hull of a precompact set is precompact. Clearly, \(\Cpt(V)\) is closed under scalar multiplication and addition. Consequently, \(t S_1 + (1-t)S_1\) is precompact whenever \(S_1\) is, for every \(t\in [0,1]\). So it suffices to show that the circled hull of a precompact set is precompact. Let \(S \subseteq F + U\), where \(U\) is a circled neighbourhood of the origin. Then the circled hull of \(S\) is contained in \(F^c + U\), where \(F^c\) is the circled hull of \(F\). Consequently, it suffices to show that \(F^c\) is precompact. Since precompact subsets are closed under finite unions, we are reduced to the case where \(F\) is a singleton \(\{x\}\). But then \(F^c\) is the image of the unit disk in \(\C\) under the (continuous) linear map \(\C \to V\), \(\lambda \mapsto \lambda x\). The conclusion now follows from the fact that the continuous image of a precompact set under a linear map is precompact. 
\end{example}


We now move to the morphism space in the world of bornologies. Unsurprisingly, we call a linear map \(T \colon V \to W\) \textit{bounded} if for every bounded subset \(S \subseteq V\), the image \(T(S)\) is bounded in \(W\). We denote the set of bounded linear maps from \(V\) to \(W\) by \(\Hom(V,W)\). 

\begin{lemma}
    Let \(L \subseteq \Hom(V,W)\) be a subset of bounded linear maps that satisfies the following: for every bounded subset \(B \subseteq V\), the set \(L(B) \defeq \setgiven{T(x)}{T \in L, x \in B}\) is bounded in \(W\). Then the collection of such \(L\) is a bornology on \(\Hom(V,W)\).
\end{lemma}
\begin{proof}
    Let \(L\) be a collection of bounded linear maps as in the hypothesis of the Lemma, and let \(T \in L^c\) be an element of the circled hull of \(L\). Then there is an \(S \in L\) and a \(\lambda \in \C\) with \(\abs{\lambda} \leq 1\) such that \(T = \lambda S\). Consequently, \(L^c(B) \subseteq L(B)\) for all bounded subsets \(B\) in \(V\), so that \(L^c\) is equibounded as well.  The same line of reasoning shows that the given collection of equibounded maps is closed under inclusions. Finite unions are trivial to check. It remains to see that the collection of equibounded linear maps is closed under taking disked hulls. Again, let \(L\) be an equibounded set of linear maps. Then its disked hull satisfies \(L^d(B) \subseteq L^d(D) \subseteq L(D)\) for any bounded subset \(B\) contained in a bounded disk \(D\).  
\end{proof}


We denote by \(\mathsf{Born}_\C\), \(\mathsf{Born}_\C^s\) and \(\mathsf{CBorn}_\C\) the categories of bornological vector spaces, separated bornological vector spaces and complete bornological vector spaces with bounded linear maps (endowed with the equibounded bornology), respectively. After we have developed some analysis, we will prove the following:

\begin{lemma}
    If \(W\) is separated (respectively, complete), then \(\Hom(V,W)\) with the equibounded bornology is separated (respectively, complete). 
\end{lemma}


\bigskip
\exs
\bigskip
\begin{exercise} \label{ex:disk-sum-iso}
Complete the proof that 
the linear map $V_{D_1} \oplus V_{D_2} / \ker(q)
\to V_{D_1+D_2}$ of Lemma \ref{lem:bornologies-directed} an isometry.
\end{exercise}

\begin{exercise} \label{ex:disk-int}
Prove that the intersection of two (complete) disks 
is again a (complete) disk.
\end{exercise}

\begin{exercise}\label{ex:int-bor}
Let $V$ be a vector space and $(\mathfrak{B}_\alpha)_{\alpha \in \Lambda}$ a
family of bornologies on $V$. 
Prove that the intersection $\bigcap_{\alpha \in \Lambda} \mathfrak B_\alpha$
is a bornology on $V$.
\end{exercise}

\begin{exercise} \label{ex:circ-conv=disk}
Prove the chain of equalities \eqref{eq:Sd=Scc}.
\end{exercise}

\begin{exercise} \label{ex:fine}
Prove that if $V$ and $W$ are 
two vector spaces equipped with
the fine bornology, then 
any linear map $T \colon V \to W$
is bounded. Moreover, show that this defines a fully faithful functor $\mathsf{Fine} \colon \mathsf{Vect}_{\C} \to \mathsf{Born}_\C$.
\end{exercise}

\begin{exercise} \label{ex:equib-op}
Let $V$ and $W$ be two Banach spaces.
Prove that the equibounded bornology
on $\Hom(V,W)$ coincides with 
the von Neumann bornology induced
by the operator norm.
\end{exercise}
\bigskip

\section{A primer on topological vector spaces}

The purpose of this section is that the reader gets comfortable with the basic objects of functional analysis.

\begin{definition}
A \emph{topological vector space} is a vector space with a topology compatible with the vector space structure. That is, addition and scalar multiplication are continuous operations. 
\end{definition}

\subsection{Spaces of continuous functions}

Let \(K\) be a compact topological space. The conitnuous functions \(f \colon K \to \C\) form a  vector space under the pointwise operations \[(f + g)(x) \defeq f(x) + g(x), \quad (\alpha \cdot f)(x) \defeq \alpha f(x).\] We denote this vector space by \(\Cont(K)\) and equip it with the norm \[\norm{f} \defeq \sup \setgiven{\abs{f(x)}}{x \in K}.\]  

\begin{proposition}\label{prop:cont-compact}
The vector space \(\Cont(K)\) with the supremum norm above is a Banach space. More generally, for any topological space \(X\), the vector space \(\Cont_b(X)\) defined by continuous, bounded functions is a Banach space for the supremum norm. 
\end{proposition}

\begin{proof}
Let \(f\), \(g \in \Cont_b(X)\) and \(\alpha \in \C\). Then for an arbitrary \(x \in X\),  \(\abs{f(x) + g(x)} \leq \abs{f(x)} + \abs{g(x)} \leq \norm{f} + \norm{g}\), so that \(\norm{f + g} \leq \norm{f} + \norm{g}\). Similarly, \(\abs{\alpha f(x)} = \abs{\lambda} \abs{f(x)}\), and taking suprema yields \(\norm{\alpha \cdot f} = \abs{\alpha} \norm{f}\). Consequently, the supremum is indeed a semi-norm. In fact, it is also a norm as if \(f \neq 0\), then \(f(x) \neq 0\) for some \(x\), so that \(\norm{f} \neq 0\). 

Next, we show that \(\Cont_b(X)\) with \(\norm{-}\) is complete. Consider a Cauchy net \((f_i)_{i \in I}\) in \(\Cont_b(X)\) for the supremum norm. For each \(x \in X\), the net \((f_i(x))_{i \in I}\) is Cauchy in \(\C\) as well, since \(\abs{f_i(x) - f_j(x)} \leq \norm{f_i - f_j}\). Since \(\C\) is complete, this net is convergent to a unique complex number \(f(x)\). \textbf{Furthermore, the net \((f_i)_{i \in I}\) converges uniformly to \(f\)}. Consequently, \(f\) is continuous. Finally, \(\norm{f} \leq \epsilon + \norm{f_i} < \infty\), so that \(f\) is bounded. This completes the proof.  
\end{proof}

  Treat examples like \(\Cont(X)\), \(\Cont_b(X)\) and \(\Cont_0(X)\). 

\subsection{Spaces of measurable functions}

For a measure space \((X,\Omega, \mu)\), treat examples like \(\mathcal{L}^p(X,\Omega,\mu)\). Specialise this to \(l^p(X)\) for the counting measure and \(\Omega\) the power-set. 

\subsection{Spaces of differentiable functions}

Treat examples like \(\Cont^\infty(M)\) and \(\mathcal{S}(\R^n)\).

\subsection{Subspaces, quotients and completions}


Let \(V\) be a semi-normed topological vector space, and \(W \subseteq V\) a subspace. Then the quotient topology on \(V/W\) is defined by the \emph{quotient semi-norm} \[\norm{x + V} \defeq \inf \setgiven{\norm{x+ w}}{w \in W}.\]

\begin{theorem}
Let \(V\) be a Banach space and \(W \subseteq V\) a closed subspace. Then \(V/W\) with the quotient norm is a Banach space. 
\end{theorem}


\begin{definition}\label{def:completion}
The \emph{completion} of a semi-normed space \(V\) is a Banach space \(\comb{V}\) with a continuous linear map \(\iota \colon V \to \comb{V}\) such that for any other continuous linear map \(f \colon V \to W\) into a Banach space, there is a unique continuous linear map \(g \colon \comb{V} \to W\) such that \(g \circ \iota = f\). 
\end{definition}


\begin{theorem}
The completion of a semi-normed topological vector space \(V\) exists and is unique. 
\end{theorem}


Completions also exist for general topological vector spaces. For our purposes, the Banach space completions considered above will suffice. 

\subsection{Bounded operators on Banach spaces}

Here's the place to talk about the category of Banach spaces.

\subsection{What exactly goes wrong?}

The category of Banach spaces is too small. Enlarge it to \emph{Frechet spaces}, but this category does not have a unique \(\Hom\)-functor. On spaces larger than Frechet spaces, there's also the issue that separate continuity is not the same as joint continuity - this leads to problems in representation theory. Finally, homological algebra is problematic in this category. 

\section{Bornological functional analysis}

At the level of Banach spaces, continuity is the same as boundedness. We enlarged this category by taking its pro-completion, but problems persist if we care about say the topology on the mapping space. How about taking its ind-completion and working instead with bounded maps? This leads to the theory of bornological vector spaces. 


\subsection{Bornological vector spaces over \(\R\) and \(\C\)}



\begin{example}[Fine bornology]
Finite dimensional subspaces of a vector space. It is the smallest bornology on a vector space.
\end{example}

\begin{example}[Von Neumann bornology]
Let \(V\) be a locally convex topological vector space. A subset \(B\subseteq V\) is \emph{von-Neumann bounded} if \(\nu(B) \subseteq \R_{>0}\) for each semi-norm \(\nu\). Recovers `usual' boundedness.  
\end{example}

\begin{example}[Pre-compact bornology]
Let \(V\) be a locally convex topological vector space. A subset \(B \subseteq V\) is \emph{pre-compact} if for every neighbourhood \(U\) of \(0\), there is a finite subset \(F \subseteq V\) such that \(B \subseteq U + F\).  
\end{example}



\begin{itemize}
\item We like to work with \emph{complete} topological or bornological vector spaces (eg: Banach spaces). Define completion for bornological vector spaces.
\item If the space is metrisable, then topological convergence and Cauchyness are equivalent to bornological convergence and Cauchyness (in the von Neumann and precompact bornologies);
\item For a metrisable space, topological completeness is equivalent to completeness in the von Neumann and precompact bornology.

\end{itemize}



\subsection{Bornological vector spaces in the nonarchimedean setting}


Copy-paste \cite{Cortinas-Cuntz-Meyer-Tamme:Nonarchimedean}*{Section 2}


\subsection{Abstract models for bornologies}


\begin{theorem}[Meyer, Meyer-Mukherjee]
The category of complete bornological vector spaces over \(\R\), \(\C\) or \(\Q_p\) embeds into the category of inductive system of Banach spaces. The essential image of this functor is the category of inductive systems of Banach spaces with injective structure maps.
\end{theorem}

Viewing complete bornological vector spaces as certain inductive systems of Banach spaces, we can therefore reinterpret the bornological theory using categorical operations on the category of Banach spaces. 

\begin{definition}
Let \(R\) be a Banach ring. The category of complete bornological \(R\)-modules is defined as the category of inductive systems of Banach \(R\)-modules with injective structure maps.
\end{definition}


\begin{theorem}
The above category has great properties.
\end{theorem}


\textbf{Mention application in analytic geometry.}