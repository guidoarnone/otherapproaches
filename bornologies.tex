\chapter{Bornologies}

\section{A primer on topological vector spaces}

The purpose of this section is that the reader gets comfortable with the basic objects of functional analysis.

\begin{definition}
A \textit{topological vector space} is a vector space with a topology compatible with the vector space structure. That is, addition and scalar multiplication are continuous operations. 
\end{definition}

\subsection{Spaces of continuous functions}

Let \(K\) be a compact topological space. The conitnuous functions \(f \colon K \to \C\) form a  vector space under the pointwise operations \[(f + g)(x) \defeq f(x) + g(x), \quad (\alpha \cdot f)(x) \defeq \alpha f(x).\] We denote this vector space by \(\Cont(K)\) and equip it with the norm \[\norm{f} \defeq \sup \setgiven{\abs{f(x)}}{x \in K}.\]  

\begin{proposition}\label{prop:cont-compact}
The vector space \(\Cont(K)\) with the supremum norm above is a Banach space. More generally, for any topological space \(X\), the vector space \(\Cont_b(X)\) defined by continuous, bounded functions is a Banach space for the supremum norm. 
\end{proposition}

\begin{proof}
Let \(f\), \(g \in \Cont_b(X)\) and \(\alpha \in \C\). Then for an arbitrary \(x \in X\),  \(\abs{f(x) + g(x)} \leq \abs{f(x)} + \abs{g(x)} \leq \norm{f} + \norm{g}\), so that \(\norm{f + g} \leq \norm{f} + \norm{g}\). Similarly, \(\abs{\alpha f(x)} = \abs{\lambda} \abs{f(x)}\), and taking suprema yields \(\norm{\alpha \cdot f} = \abs{\alpha} \norm{f}\). Consequently, the supremum is indeed a semi-norm. In fact, it is also a norm as if \(f \neq 0\), then \(f(x) \neq 0\) for some \(x\), so that \(\norm{f} \neq 0\). 

Next, we show that \(\Cont_b(X)\) with \(\norm{-}\) is complete. Consider a Cauchy net \((f_i)_{i \in I}\) in \(\Cont_b(X)\) for the supremum norm. For each \(x \in X\), the net \((f_i(x))_{i \in I}\) is Cauchy in \(\C\) as well, since \(\abs{f_i(x) - f_j(x)} \leq \norm{f_i - f_j}\). Since \(\C\) is complete, this net is convergent to a unique complex number \(f(x)\). \textbf{Furthermore, the net \((f_i)_{i \in I}\) converges uniformly to \(f\)}. Consequently, \(f\) is continuous. Finally, \(\norm{f} \leq \epsilon + \norm{f_i} < \infty\), so that \(f\) is bounded. This completes the proof.  
\end{proof}

  Treat examples like \(\Cont(X)\), \(\Cont_b(X)\) and \(\Cont_0(X)\). 

\subsection{Spaces of measurable functions}

For a measure space \((X,\Omega, \mu)\), treat examples like \(\mathcal{L}^p(X,\Omega,\mu)\). Specialise this to \(l^p(X)\) for the counting measure and \(\Omega\) the power-set. 

\subsection{Spaces of differentiable functions}

Treat examples like \(\Cont^\infty(M)\) and \(\mathcal{S}(\R^n)\).

\subsection{Subspaces, quotients and completions}


Let \(V\) be a semi-normed topological vector space, and \(W \subseteq V\) a subspace. Then the quotient topology on \(V/W\) is defined by the \textit{quotient semi-norm} \[\norm{x + V} \defeq \inf \setgiven{\norm{x+ w}}{w \in W}.\]

\begin{theorem}
Let \(V\) be a Banach space and \(W \subseteq V\) a closed subspace. Then \(V/W\) with the quotient norm is a Banach space. 
\end{theorem}


\begin{definition}\label{def:completion}
The \textit{completion} of a semi-normed space \(V\) is a Banach space \(\comb{V}\) with a continuous linear map \(\iota \colon V \to \comb{V}\) such that for any other continuous linear map \(f \colon V \to W\) into a Banach space, there is a unique continuous linear map \(g \colon \comb{V} \to W\) such that \(g \circ \iota = f\). 
\end{definition}


\begin{theorem}
The completion of a semi-normed topological vector space \(V\) exists and is unique. 
\end{theorem}


Completions also exist for general topological vector spaces. For our purposes, the Banach space completions considered above will suffice. 

\subsection{Bounded operators on Banach spaces}

Here's the place to talk about the category of Banach spaces.

\subsection{What exactly goes wrong?}

The category of Banach spaces is too small. Enlarge it to \textit{Frechet spaces}, but this category does not have a unique \(\Hom\)-functor. On spaces larger than Frechet spaces, there's also the issue that separate continuity is not the same as joint continuity - this leads to problems in representation theory. Finally, homological algebra is problematic in this category. 

\section{Bornological functional analysis}

At the level of Banach spaces, continuity is the same as boundedness. We enlarged this category by taking its pro-completion, but problems persist if we care about say the topology on the mapping space. How about taking its ind-completion and working instead with bounded maps? This leads to the theory of bornological vector spaces. 


\subsection{Bornological vector spaces over \(\R\) and \(\C\)}

Throughout this section, we work with vector spaces over \(\R\) and \(\C\).

\begin{definition}[Bornology]
A \textit{bornology} on a vector space \(V\) is a collection \(\mathfrak{P}\) of its subsets (called \textit{bounded subsets}) satisfying the following natural properties:

\begin{itemize}
\item \(\{x\} \in \mathfrak{P}\) for all \(x \in V\);
\item if \(S \subseteq T\) and \(T \in \mathfrak{P}\), then \(S \in \mathfrak{P}\);
\item \(S \cup T \in \mathfrak{P}\) whenever \(S\), \(T \in \mathfrak{P}\);
\item if \(S \in \mathfrak{P}\), then \(r S \in \mathfrak{P}\) for \(r>0\);
\item any subset \(S \in \mathfrak{P}\) is contained in a disk \(T \in \mathfrak{P}\). 
\end{itemize}
 
\end{definition}



\begin{example}[Fine bornology]
Finite dimensional subspaces of a vector space. It is the smallest bornology on a vector space.
\end{example}

\begin{example}[Von Neumann bornology]
Let \(V\) be a locally convex topological vector space. A subset \(B\subseteq V\) is \textit{von-Neumann bounded} if \(\nu(B) \subseteq \R_{>0}\) for each semi-norm \(\nu\). Recovers `usual' boundedness.  
\end{example}

\begin{example}[Pre-compact bornology]
Let \(V\) be a locally convex topological vector space. A subset \(B \subseteq V\) is \textit{pre-compact} if for every neighbourhood \(U\) of \(0\), there is a finite subset \(F \subseteq V\) such that \(B \subseteq U + F\).  
\end{example}



\begin{itemize}
\item We like to work with \textit{complete} topological or bornological vector spaces (eg: Banach spaces). Define completion for bornological vector spaces.
\item If the space is metrisable, then topological convergence and Cauchyness are equivalent to bornological convergence and Cauchyness (in the von Neumann and precompact bornologies);
\item For a metrisable space, topological completeness is equivalent to completeness in the von Neumann and precompact bornology.

\end{itemize}



\subsection{Bornological vector spaces in the nonarchimedean setting}


Copy-paste \cite{Cortinas-Cuntz-Meyer-Tamme:Nonarchimedean}*{Section 2}


\subsection{Abstract models for bornologies}


\begin{theorem}[Meyer, Meyer-Mukherjee]
The category of complete bornological vector spaces over \(\R\), \(\C\) or \(\Q_p\) embeds into the category of inductive system of Banach spaces. The essential image of this functor is the category of inductive systems of Banach spaces with injective structure maps.
\end{theorem}

Viewing complete bornological vector spaces as certain inductive systems of Banach spaces, we can therefore reinterpret the bornological theory using categorical operations on the category of Banach spaces. 

\begin{definition}
Let \(R\) be a Banach ring. The category of complete bornological \(R\)-modules is defined as the category of inductive systems of Banach \(R\)-modules with injective structure maps.
\end{definition}


\begin{theorem}
The above category has great properties.
\end{theorem}


\textbf{Mention application in analytic geometry.}