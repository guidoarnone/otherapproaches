\chapter*{Introduction}
\addcontentsline{toc}{chapter}{Introduction}

Functional analysis is the generalisation of linear algebra to infinite dimensions. What this more concretely means is that instead of working with operators on finite-dimensional vector spaces over a field, we work with linear operators on a possibly infinite-dimensional vector space. To make sense of such operators, we can no longer treat infinite-dimensional vector spaces as algebraic objects, but rather as topological objects. This necessitates the study of topological vector spaces and topological algebras of operators on such spaces. In this course, we shall unravel certain foundational difficulties in working with categories of such spaces - this will essentially entail replacing topology with a somewhat different axiomatic treatment of analysis (that still contains the original objects we were interested in). 

Before moving into the technical details, let us briefly illustrate some of the difficulties that arise in the theory of topological vector spaces. Given two ordinary vector spaces \(X\) and \(Y\), we can perform two canonical operations on them -- the \emph{tensor product} \(X \otimes Y\) and the \emph{internal Hom} \(\Home(X,Y)\). These are both vector spaces related by the familiar duality \[\Hom(X \otimes Y, Z) \cong \Hom(X, \Home(Y,Z))\] called the \emph{tensor-Hom adjunction}. This leads to several interesting consequences, such as the commutativity of the tensor product with inductive limits. It is consequently desirable to have such a tensor-Hom adjunction even in the topological setting. And this is where the trouble begins. Given two complete topological vector spaces, one can form their \emph{completed, projective (topological) tensor product}. This is defined as the universal object for which jointly continuous bilinear maps \(X \times Y \to Z\) into an arbitrary complete, topological vector space  extend to  continuous linear maps \(X \hot_{\pi} Y \to Z\). However, it has no right adjoint functor, as if it did, it would commute with colimits which is false. 

Next, if we replace joint continuity of bilinear maps with separate continuity in each variable, we arrive at the so-called \emph{inductive tensor product} \(X \hot_\iota Y\). The resulting functor \(- \hot_\iota Y\) \emph{does} commute with inductive limits. However, the injective tensor product is not associative. Another possible attempt to remedy this is to work in the larger category of \emph{all} (and not just complete) locally convex topological vector spaces. In this category, we do have an associative tensor product, namely, the incomplete inductive tensor product, which is left adjoint to the Hom functor, where the mapping space between two such vector spaces is equipped with the topology of pointwise convergence. But this category is too unwieldy to be practical. 

Let us now revisit the smallest useful category, where the problems above do not arise. This is the category \(\mathsf{Ban}\) of Banach spaces with bounded linear maps as morphisms. In this category, we do have an internal Hom given by the Banach space \(\Home(X,Y)\) of bounded linear maps with the operator norm \(\norm{T} \defeq \sup_{\norm{x}_X = 1} \norm{T(x)}_W\), and summarily a  tensor-Hom adjunction \[\Hom(V \hot_\pi W, Z) \cong \Hom(V, \Home(W,Z))\] relative to the completed projective tensor product. The category of Banach spaces is closed under finite limits and colimits, but not under infinite ones. Historically, the way out of this has been to work in the category \(\mathsf{Pro}(\mathsf{Ban})\) of formal projective systems of Banach spaces. This is a category with all limits and colimits, but it still lacks an internal Hom operation.  


This course visits two alternatives to the problems illustrated above. One alternative is to consider the category \(\mathsf{Ind}(\mathsf{Ban})\) of inductive systems of Banach spaces. Just like the pro-completion, the category of inductive systems of Banach spaces is closed under all limits and colimits, but additionally, it also has an internal \(\Home\)-functor that implements the tensor-Hom adjunction we were after. Furthermore, this category is also ideal for homological algebra, as it is almost an abelian category -- it is \emph{quasi-abelian}.  

Another approach is topos-theoretic. More concretely, one tries to embed the category of topological vector spaces into a suitable presheaf category \(\mathsf{Fun}(\mathsf{CHaus}^\op, \mathsf{Set})\). Taking sheaves with respect to the site where covers are given by surjections of compact Hausdorff spaces, we obtain the topos of \emph{condensed vector spaces}.  And being a topos, it contains all limits and colimits. This category contains all reasonable classes of topological vector spaces, and is additionally also an abelian category (of the nicest kind, namely, a \emph{Grothendieck abelian category}).



Having hopefully convinced the reader of the need to look deeper into the foundations of topological vector spaces, we also mention that the approaches that will be discussed in this course go far beyond ``bug-fixing" in classical theories. Some examples in support of the novelty of the approaches are as follows:

\begin{itemize}
\item To prove important results about cyclic homology and \(K\)-theory of topological algebras, one needs appropriate notions of \emph{topological nilpotence} and ``smooth'' subalgebras of Banach algebras. Roughly speaking, these  definitions are implemented by ensuring that certain power series of noncommuting variables converge inside some bounded subset. This requires a notion of spectral radii of bounded subsets (rather than single elements, which the reader may be familiar with from a course on functional analysis).

\item In recent work on cyclic homology, we introduce a class of algebras that are defined as certain generalised completions of algebras without any topology. These  completions have no convenient, purely topological description. 

\item It is a convenient property of a cohomology theory to send kernel-cokernel pairs in its domain to long-exact sequences of abelian groups. This happens for instance for short exact sequences of topological algebras when the cohomology theory in question is, say, an appropriate variant of periodic cyclic homology. It is desirable to have a similar property for continuous group cohomology for topological groups acting on topological vector spaces. Again, a straightforward treatment of homological algebra does not offer anything on this front.  
\end{itemize}